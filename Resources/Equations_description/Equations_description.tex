\documentclass{article}
\usepackage{amsmath}
\usepackage{amssymb}
\usepackage{bm}
\usepackage{hyperref}
\usepackage{csquotes}

\DeclareSymbolFont{matha}{OML}{txmi}{m}{it}% txfonts
\DeclareMathSymbol{\varv}{\mathord}{matha}{118}

\title{Macroeconomics 2 Presentation \\ Equations description of \\ \textquote{A Behavioral New Keynesian Model} by Xavier Gabaix}
\author{Gugelmo Cavalheiro Dias Paulo \\ \and Mitash Nayanika \\ \and Wang Shang}
\date{\today}
\hypersetup{
    colorlinks,
    citecolor=black,
    filecolor=black,
    linkcolor=black,
    urlcolor=black
}

\begin{document}

\maketitle

This file aims to summarise and give some all the equations of the presented article. Its goal is to serve as a complement to the presentation of May 3, 2024 in the Macroeconomics 2 class. 
Although we use the same numerisation of the equations as in the article, we do present them in our own structure, following the oral presentation divided in three main sections. 
The sections of the article are however given as subsections, to provide some additional information for those wanting to use this document as a complementary helper to the reading of the article.

\pagebreak
\tableofcontents
\pagebreak

\section{A Behavioral Model}
The first two equations are the same as (28) and (29) and will be explained later.

\subsection{Introduction}

\subsubsection*{Equation 1}
% equation 1%
\begin{equation}
    x_{t}=M\cdot\mathbb{E}_{t}\left[ x{t+1} -\sigma (i_{t}-\mathbb{E}_{t}\left[\pi_{t+1}\right]-r^{n}_{t})\right]
\end{equation}

\subsubsection*{Equation 2}
% equation 2%
\begin{equation}
    \pi_{t}=\beta\cdot M^{f} \mathbb{E}_{t}\left[\pi_{t+1}\right]+\kappa\cdot x_{t}
\end{equation}

\subsection{Basic Setup and the Household’s Problem}

\subsubsection*{Equation 3}
% equation 3%
\begin{equation}
    U = \mathbb{E}_{t} \left[ \sum{t=0}^{\infty} \beta^{t}\left(\frac{c_{t}^{1-\gamma}-1}{1-\gamma} - \frac{N_{t}^{1+\phi}}{1+\phi}\right)\right]
\end{equation}

Equation (3) is just the flow utility of the Household, with : 
\begin{itemize}
    \item $\beta$ the discount factor
    \item $c_{t}$ the consumption of the houshold at time $t$
    \item $N_{t}$ the work of the household at time $t$
    \item $\gamma$ determines the concavity of the utility function with respect to the consumption, i.e. the importance of consumption in the utility function
    \item $\phi$ determines the concavity of the utility function with respect to work, i.e. the importance of work in the utility function
\end{itemize}

%There is some computation between (3) and (4) !!!%

\subsubsection*{Equation 4}
% equation 4%
\begin{equation}
    k_{t+1}=(1+r_t)(k_{t}-c_{t}+y_{t})
\end{equation}

Equation (4) is the law of motion of the real financial wealth of the household, where : 
\begin{itemize}
    \item $k_t$ is the real financial welath of the household at time $t$
    \item $r_t$ is the real interest rate
    \item $w_t$ is the real wage
    \item $y_t$ is the agent's real income, defined as $y_t=w_t\cdot N_{t}+y_{t}^{f}$, with $y_{t}^{f}$ the profit income (or the income from firms) at time $t$
\end{itemize}

\subsubsection*{Equation 5}

% equation 5%
\begin{equation}
    \bm{X}_{t+1}=\bm{G}^{\bm{X}}\left(\bm{X}_{t},\bm{\epsilon}_{t+1}\right)
\end{equation}

Equation (5) describes the evolution of macroeconomic variables, where : 
\begin{itemize}
    \item $\bm{X}_t$ is the state vector, including several macroeconomic variables of time $t$, like $\zeta_t$ the aggregate TFP, and the announced actions in monetary and fiscal policy
    \item $\bm{G}^X$ the equilibrium transition function, i.e. the function that gives the macroeconomic variables at time $t+1$ from the macroeconomic variables at the previous period
    \item $\epsilon_{t}$ is the innovation in the economy at time $t$, with $\mathbb{E}_{t}\left[\epsilon{t+1}\right]=0$, that depends on the equilibrium policies of the agent and of the government
\end{itemize}

\subsubsection*{Equation 6}

% equation 6%
\begin{equation}
    k_{t+1}=G^{k}(c_{t},N_{t}, k_{t}, \bm{X}_{t}):= (1+\bar{r}+\hat{r}(\bm{X}_{t}))(k_{t}+\bar{y}+\hat{y}_(N_{t},\bm{X}_{t})-c_{t})
\end{equation}

Equation (6) is the application of the consideration of a set of macroeconomic variables on the law of motion of real financial wealth $k_{t}$, where : 
\begin{itemize}
    \item $\bar{r}$ is the steady state value of the real interest rate, that does not depend on time
    \item $\hat{r}(\bm{X}_{t})$ is the value of the deviation from the steady state of the real interest rate, that depends on the state vector $\bm{X}_{t}$ at time $t$
    \item $\bar{y}$ is the steady state value of the agent's real income, that does not depend on time
    \item $\hat{y}(N_{t},\bm{X}_{t})$ is the deviation from the steady state of the agent's real income, that depends on the number of hours worked at time $t$ and on the state vector at time $t$
    \item $c_{t}$ is the aggregate consumption level at time $t$ of the agent
\end{itemize}

\subsubsection*{Equation 7}

% equation 7%
\begin{equation}
    \bm{X}_{t+1}=\bm{\Gamma}\bm{X}_{t}+\bm{\epsilon}_{t+1}
\end{equation}

Equation (7) describes the linear version of the equilibrium transition function, it is the linearization of the law of motion, where :
\begin{itemize}
    \item $\Gamma$ is a squared matrix that multiplies the state vector
    \item $\bm{X}_t$ is the state vector at time t
    \item $\epsilon_{t}$ is the innovation shock
\end{itemize}

\subsubsection*{Equation 8 (Assumption 1)}

% equation 8%
\begin{equation}
    \bm{X}_{t+1}=\bar{m}\cdot\bm{G}^{\bm{X}}(\bm{X}_{t},\bm{\epsilon}_{t+1})
\end{equation}

Equation (8) describes the Cognitive Discounting of the State Vector, i.e. the perception by behavioural agents of the law of motion of the macroeconomic variables, where : 
\begin{itemize}
    \item $\bar{m}\in\left[0n1\right]$ is the cognitive discount factor measuring the attention to the future
\end{itemize}

\subsubsection*{Equation 9}

% equation 9%
\begin{equation}
    \bm{X}_{t+1}=\bar{m}(\bm{\Gamma}\bm{X}_{t}+\bm{\epsilon}_{t+1})
\end{equation}

Equation (9) is just the linearized version of the perception by behavioral agents of the law of motion of the state vector. 

\subsubsection*{Equation 10}

\begin{equation}
    \mathbb{E}_{t}^{BR}\left[\bm{X}_{t+k}\right]=\bar{m}^{k}\mathbb{E}_{t}\left[\bm{X}_{t+k}\right]
\end{equation}

Equation (10) defines the expectation of behavioral agents in function of the rational perception of the law of motion of the state vector, where : 
\begin{itemize}
    \item $k\geq 0$ a time period in discrete context
    \item $\mathbb{E}_{t}^{BR}\left[\bm{X}_{t+k}\right]$ is the expected value of the state vector at time $t+k$ by behavioral agents (or subjective/behavioral expectation operator)
    \item $\bar{m}^{k}$ is the cognitive discounting effect at period $t+k$
    \item $\mathbb{E}_{t}\left[\bm{X}_{t+k}\right]$ is the rational expectation of the state vector at time $t+k$
\end{itemize}

\subsubsection*{Equation 11 (Lemma 1)}

\begin{equation}
    \mathbb{E}_{t}^{BR}\left[z\left(\bm{X}_{t+k}\right)\right]=\bar{m}^{k}\mathbb{E}_{t}\left[z\left(\bm{X}_{t+k}\right)\right]
\end{equation}

Equation (11) defines in the general case the behavioral expectation operator, for any function of the state vector, where : 
\begin{itemize}
    \item $k\geq 0$ a time period in discrete context
    \item $z(\cdot)$ is a function, such that $z(0)=0$
    \item $\mathbb{E}_{t}^{BR}\left[z\left(\bm{X}_{t+k}\right)\right]$ is the expected value of the image of the state vector by the function $z(\cdot)$ at time $t+k$ by behavioural agents
    \item $\bar{m}^{k}$ is the cognitive discounting effect at period $t+k$
    \item $\mathbb{E}_{t}\left[z\left(\bm{X}_{t+k}\right)\right]$ is the rational expectation of the image of the state vector by the function $z(\cdot)$ at time $t+k$
\end{itemize}

\subsubsection*{Equation 12}

\begin{equation}
    \mathbb{E}_{t}^{BR}\left[\bar{r}+\hat{r}\left(\bm{X}_{t+k}\right)\right]=\bar{r}+\bar{m}^{k}\mathbb{E}_{t}\left[\hat{r}(\bm{X}_{t+k})\right]
\end{equation}

Equation (12) is an example of the Lemma 1 applied to the interest rate, where :
\begin{itemize}
    \item $k\geq 0$ a time period in discrete context
    \item $\bar{r}$ the steady state level of the real interest rate, that does not depend on time,  
    \item $\hat{r}(\bm{X}_{t+k})$ is the equilibrium transition function defining the value of the deviation from the steady state of the real interest rate in the function of the state vector at time $t+k$
    \item $\bar{r}+\hat{r}(\bm{X}_{t})=r{t}(\bm{X}_{t})$ is the value of the real interest rate at time $t$
    \item $\mathbb{E}_{t}^{BR}\left[\bar{r}+\hat{r}(\bm{X}_{t+k})\right]$ is the expected value of the real interest at time $t+k$ by behavioural agents
    \item $\mathbb{E}_{t}\left[\hat{r}(\bm{X}_{t+k})\right]$ is the rational expectation of the value of the deviation of the real interest rate from the steady state at time $t+k$
\end{itemize}

\subsection{The Firm's problem}

\subsubsection*{Equation 13}
\begin{equation}
    P_{t}=\left(\int_{0}^{1}P_{it}^{1-\varepsilon}\,di\right)^{\frac{1}{1-\varepsilon}}
\end{equation}

Equation (13) describes the aggregate price level, where : 
\begin{itemize}
    \item $P_{t}$ is the aggregate price level of the economy at time $t$
    \item $i\in\left[0,1\right]$ is the firm index
    \item $\varepsilon$ is the elasticity of substitution between goods
\end{itemize}

\subsubsection*{Equation 14}
\begin{equation}
    \varv^{0}(q_{i\tau},\mu_{\tau},c_{\tau}):=\left(e^{q_{i\tau}}-(1-\tau_{f})e^{-\mu_{\tau}}\right)e^{-\varepsilon q_{i\tau}}c_{\tau}
\end{equation}

Equation (14) describes the profit of the firm before the lump sum tax of the government, where : 
\begin{itemize}
    \item $\varv$ is the real profit of the firm
    \item $q_{i\tau}=\text{ln}\left(\frac{P_{i\tau}}{P_{\tau}}\right)=p_{i\tau}-p_{\tau}$ is the real log price at time $\tau$
    \item $\tau_f=\frac{1}{\varepsilon}$ it the corrective wage subsidy from the government, funded by the lump sum tax
    \item $\mu_{\tau}=\zeta_{t}-\text{ln}(\omega_{t})$ is the labor wedge, which is zero at efficiency
    \item $\varepsilon$ is the elasticity of subsititution between goods
    \item $c_{\tau}$ is the aggregate level of consumption
\end{itemize}

\subsubsection*{Equation 15}
\begin{equation}
    \varv\left(q_{it},\bm{X}{\tau}\right):=\varv^{0}\left(q_{it}-\Pi(\bm{X}{\tau}),\mu(\bm{X}{\tau}),c(\bm{X}_{\tau})\right)
\end{equation}

Equation 15 describes the flow profit of the firm in function of the real log price and of the extended macro state vector, where : 
\begin{itemize}
    \item $q_{it}=\text{ln}\left(\frac{P_{it}}{P_{t}}\right)=p_{it}-p_{t}$ is the real log price
    \item $\bm{X}{\tau}=(\bm{X}^{\mathcal{M}}{\tau}, \Pi_{\tau})$ is the extended macro state vector, with $\bm{X}^{\mathcal{M}{\tau}}$ the vector of macro variables, including $\zeta{\tau}$ and possible announcements about future policy
    \item $\Pi\left(\bm{X}{\tau}\right):=p{\tau}-p_{t}=\pi_{t+1}+...+\pi_{\tau}$ is the inflation between times $t$ and $\tau$
    \item $q_{it}-\Pi(\bm{X}{\tau})=q{i\tau}$ is the real price of the firm if they didn't change its price between $t$ and $\tau$
    \item $\mu\left(\bm{X}_{\tau}\right)$ is the aggregate labor wedge in function of the extended state vector at time $t$
    \item $c(\bm{X}_{\tau})$ is the aggregate consumption level in function of the extended state vector at time $t$
\end{itemize}

\subsubsection*{Equation 16}
\begin{equation}
\max_{\text{q}{it}} \mathbb{E}_{t} \sum_{\tau=t}^{\infty}(\beta\theta)^{\tau-t} \frac{c\left(\bm{X_{\tau}}\right)^{-\gamma}}{\left(\bm{X_{t}}\right)^{-\gamma}} \varv\left(q_{it},\bm{X}{\tau}\right)
\end{equation}

Equation (16) describes the maximisation program of the firm given that they have a Calvo-like probability of $\theta$ of being able to change their price at each period, where : 

\begin{itemize}
    \item $t$ is the initial period 
    \item $\tau$ is the time period index  
    \item $q_{it}$ is the real log price of the firm at time $t$
    \item $\beta$ is the discount factor
    \item $\theta$ is the Calvo like probability that the firm can change its price at any period
    \item $\frac{c\left(\bm{X}{\tau}^{-\gamma}\right)}{c\left(\bm{X}_{t}^{-\gamma}\right)}$ is the adjustment in the stochastic discount factor/pricing kernel between times $t$ and $\tau$ 
\end{itemize}

\subsubsection*{Equation 17}
\begin{equation}
    \max_{q_{it}}{\mathbb{E}_{t}^{BR} \left[\sum{\tau=t}^{\infty}(\beta\theta)^{\tau-t}\frac{c\left(\bm{X}{\tau}^{-\gamma}\right)}{c\left(\bm{X}_{t}^{-\gamma}\right)} \varv\left(q_{it,\bm{X}_{\tau}}\right) \right]}
\end{equation}

Equation 17 describes the maximisation program of the behavioral firm, i.e. such that it is maximisation the behavioral expectation operator of the flow profit, where : 
\begin{itemize}
    \item $\mathbb{E}_{t}^{BR}$ is the behavioral/subjective expected value operator 
    \item $t$ is the initial period 
    \item $\tau$ is the time period index  
    \item $q_{it}$ is the real log price of the firm at time $t$
    \item $\beta$ is the discount factor
    \item $\theta$ is the Calvo like probability that the firm can change its price at any period
    \item $\frac{c\left(\bm{X}{\tau}^{-\gamma}\right)}{c\left(\bm{X}_{t}^{-\gamma}\right)}$ is the adjustment in the stochastic discount factor between times $t$ and $\tau$ and it is approximately 1 when linearised around deterministic steady state. 
\end{itemize}

\subsection{Model solution}

\subsubsection*{Equation 18}
\begin{equation}
    \hat{c}_{t}=\mathbb{E}_{t}\left[\hat{c}_{t+1}-\frac{1}{\gamma R}\hat{r}_{t}\right]
\end{equation}

Equation (18) is the linearized version of the Euler equation obtained from the presented model. It is also called the investment-savings (IS) curve, where : 
\begin{itemize}
    \item $\hat{c}_{t}$ is the value of the deviation from the steady state of the aggregate consumption at time $t$
    \item $\mathbb{E}_{t}\left[\hat{c}_{t+1}\right]$ is the rational expectation of the value of the deviation from the steady state of the aggregate consumption at time $t+1$
    \item $\gamma$ is the factor of the importance of consumption
    \item $R:=1+\bar{r}$ is defined from the real interest rate at the steady state (cf. page 7 of the article)
\end{itemize}

\subsubsection*{Equation 19}
\begin{equation}
    \hat{c}_{t}=M\cdot\mathbb{E}_{t}\left[\hat{c}_{t+1}\right]-\sigma\hat{r}_{t}
\end{equation}

Equation (19) is the application of Lemma 1 (equation (11)) on the previous Euler equation, i.e. a cognitively discounted aggregate Euler equation, where : 
\begin{itemize}
    \item $M$ is the macro parameter of attention, such that $M=\bar{m}$ here
    \item $\sigma=\frac{1}{\gamma R}$
\end{itemize}

\subsubsection*{Equation 20}
\begin{equation}
    N^{\phi}_{t}=\omega_{t}c_{t}^{\gamma}
\end{equation}

Equation (20) the result of the static First Order Condition for labor supply, where : 
\begin{itemize}
    \item $N_{t}$ is the quantity of labor provided at time $t$
    \item $\omega_{t}$ is the real wage at time $t$
    \item $c_{t}$ is the aggregate quantity of consumption at time $t$
    \item $\gamma$ is the consumption importance in the utility
\end{itemize}

\subsubsection*{Equation 21}
\begin{equation}
    \hat{c}_{t}^{n}=\frac{1+\phi}{\gamma+\phi}\zeta{t}
\end{equation}

Equation (21) is the equation giving us the consumption in a natural economy where there are no frictions in prices i.e. flexible price economy. where :
\begin{itemize}
    \item $\hat{c}_{t}^{n}$ is the flexible price/natural economy consumption 
    \item $\gamma$ is the the elasticity of substitution between goods. 
    \item $\phi$ is the Frisch elasticity of labour
    \item $\zeta_{t}$ is the Total Factor prodcutivity

We calculate this natural economy consumption the same way we calculated the flexible price economy consumption using the log-linearised version of the natural economy price, first-order condition for labour supply, and market clearing condition for consumption and income. 
\end{itemize} is the flexible price/natural economy consumption

\subsubsection*{Equation 22}
\begin{equation}
    \hat{c}^{n}_{t} = M\cdot\mathbb{E}_{t}\left[\hat{c}^{n}_{t+1}\right]-\sigma\hat{r}^{n}_{t}
\end{equation}

Equation (22) is the application of Lemma 1 (equation (11)) on the previous Euler equation, i.e. a cognitively discounted aggregate Euler equation for the natural economy where there are no price frictions. Here : 
\begin{itemize}
    \item $M$ is the macro parameter of attention, such that $M=\bar{m}$ here
    \item $\sigma=\frac{1}{\gamma R}$ 
    \item $\hat r_{t}^{n}$ is the real interest rate for the natural economy
\end{itemize}

\subsubsection*{Equation 23}
\begin{equation}
    r^{n0}_{t}=\bar{r}+\frac{1+\phi}{\sigma(\gamma+\phi)}\left(M\cdot\mathbb{E}_{t}\left[\zeta_{t+1}\right]-\zeta_{t}\right)
\end{equation}
Equation (23) is the equation for the pure natural rate of interest: this is the interest rate that prevails in an economy without pricing frictions, and undisturbed by government policy (in particular, budget deficits). In this equation : 
\begin{itemize}
    \item $r^{n0}_{t}$ is pure natural rate of interest
    \item $\gamma$ is the the elasticity of substitution between goods. 
    \item $\phi$ is the Frisch elasticity of labour
    \item $\sigma=\frac{1}{\gamma R}$ 
    \item $M$ is the macro parameter of attention, such that $M=\bar{m}$ here
    \item $\zeta_{t}$  and $\zeta_{t+1}$ is the Total Factor productivity in time $t$ and $t+1$ 
        respectively
\end{itemize}
    
This pure natural interest rate is calculated by isolating the $\hat r_{t}^{n}$ in Equation (22) and then replacing $\hat{c}_{t}^{n}=\frac{1+\phi}{\gamma+\phi}\zeta{t}$ , $\hat{c}_{t+1}^{n}=\frac{1+\phi}{\gamma+\phi}\zeta{t+1}$ and $\hat r_{t}^{n}=r_{t}^{n}-\bar{r}$ and by assuming $r_{t}^{n}=r^{n0}_{t}$ which holds when there are no budget deficits. 

\subsubsection*{Equation 24}
\begin{equation}
    x_{t}=M\cdot\mathbb{E}_{t}\left[x{t+1}\right]-\sigma(\hat{r}_{t}-\hat{r}^{n}_{t})
\end{equation}

Equation (24) is the equation for the behavioural discounted Euler equation where : 
\begin{itemize}
    \item $x_{t}$ is the output gap at period $t$
    \item $x_{t+1}$ is the output gap at period $t+1$
    \item $M$ is the macro parameter of attention, such that $M=\bar{m}$ here
    \item $\sigma=\frac{1}{\gamma R}$ 
    \item $\hat r_{t}$ is the deviation of the real interest rate from the steady state.
    \item $\hat r_{t}^{n}$ is the deviation of the real interest rate from the steady state for the natural economy.
\end{itemize}

This is calculated by subtracting the expression $\hat{c}_{t}=M\cdot\mathbb{E}_{t}\left[\hat{c}_{t+1}\right]-\sigma\hat{r}_{t}$ i.e. Equation (19) and $\hat{c}^{n}_{t} = M\cdot\mathbb{E}_{t}\left[\hat{c}^{n}_{t+1}\right]-\sigma\hat{r}^{n}_{t}$ i.e. Equation (22)

\subsubsection*{Equation 25}
\begin{equation}
    x_{t}=M\cdot\mathbb{E}_{t}\left[x{t+1}\right]-\sigma(i_{t}-\mathbb{E}_{t}\left[\pi_{t+1}\right]-r^{n}_{t})
\end{equation}

Equation (25) is the behavioural discounted Euler equation after replacing the equation for Fisher equation  where $\hat r_{t}= r_{t}- \bar r = (i_{t}-\mathbb{E}_{t}\left[\pi_{t+1}\right]-r^{n}_{t})$
\begin{itemize}
    \item $i_{t}$ is the nominal interest rate
    \item $\mathbb{E}_{t}\left[\pi_{t+1}\right]$ is the expected inflation in the future
    \item $M=\bar{m}$
    \item $\sigma=\frac{1}{\gamma R}$
\end{itemize}

\subsubsection*{Equation 26}
\begin{equation}
    x_{t}=-\sigma\sum_{k\geq 0}{M\cdot \mathbb{E}_{t}\left[\hat{r}_{t+k}-\hat{r}_{t+k}^{n}\right]}
\end{equation}

Equation (26) iteratively using Equation (24) reduces to this equation. It indicates that changes in the interest rate in the $1000^{th}$ period will have a discounted impact on the output gap. Therefore, the effect of changes in interest rates on the output gap diminishes over time, with changes in the $1000^{th}$ period having a smaller impact compared to changes in the $1^{st}$ period.


\subsubsection*{Equation 27}
\begin{equation}
    p^{*}_{t}=p_{t}+(1-\beta\theta)\sum_{k=0}^{\infty}\left(\beta\theta\bar{m}\right)^{k}\cdot\mathbb{E}_{t}\left[\pi_{t+1}+...+\pi_{t+k}-\mu_{t+k}\right]
\end{equation}

Equation (27) is the optimal price for a behaviourial firm resetting its price
\begin{itemize}
    \item $p^{*}_{t}$ is the the price a behaviourial firm will reset its price to. 
    \item $\bar{m}$ is the cognitive discounting factor. 
    \item $p^{*}_{t}= q{it}+p_{t}$ where $q_{it}$ is the linearised version of the maximisation solution to problem 17. $q_{it}$ is the price for the firms that can adjust their prices and $p_{t}$ is the price of the firms that can't adjust their prices.
\end{itemize}

\subsubsection*{Equation 28 - Proposition 2, the first equation}
\begin{equation}
    x_{t}=M\cdot\mathbb{E}_{t}\left[x{t+1}\right]-\sigma(i_{t}-\mathbb{E}_{t}\left[\pi_{t+1}\right]-r^{n}_{t})
\end{equation}

Equation (28) is the Behavioural IS Curve. Essentially we are representing the Euler equation in terms of the output gap and also using the Fisher Equation.

\subsubsection*{Equation 29 - Proposition 2, second equation}
\begin{equation}
    \pi_{t}=\beta\cdot M^{f} \mathbb{E}_{t}\left[\pi_{t+1}\right]+\kappa\cdot x_{t}
\end{equation}

Equation (29) is the Behavioural Phillips curve where: 
\begin{itemize}
    \item $M^{f}$ is the aggregate attention parameter for firms and $M^{f}=\bar{m}\left(\theta+\frac{1-\beta\theta}{1-\beta\theta\bar{m}}(1-\theta)\right)$
    \item $\kappa=\widetilde{\kappa}$  is the slope of the Phillips curve and $\widetilde{\kappa} =  \frac{(1-\theta)(1-\beta\theta)}{\theta}(\gamma+\phi)$ which is the slope obtained from fully rational firms. 

\end{itemize}

\subsubsection*{Equation 30}
\begin{equation}
    \begin{cases}
        M=\bar{m} \\
        \sigma=\frac{1}{\gamma R} \\
        M^{f}=\bar{m}\left(\theta+\frac{1-\beta\theta}{1-\beta\theta\bar{m}}(1-\theta)\right)
    \end{cases}
\end{equation}

Equation (30) defines the parameters in Proposition 2.


\section{Consequences of the Model}

\begin{equation}
    A = B
\end{equation}

This is a description


\section{Behavioral Enrichments to the Model}

\subsection{Term Structure of Consumer Attention}

\subsubsection*{Equation 49}

\begin{equation} \tag{49}
    \begin{split}
        k_{t+1}= &\  \textbf{G}^{k,BR}(c_{t},N_{t},k_{t},\textbf{X}_{t}) \\
        & := (1+\bar{r}+\hat{r}^{BR}(\textbf{X}_t))(k_{t}+\bar{y}+\hat{y}^{BR}(N_{t},\textbf{X}_t)-c_{t})
    \end{split}
\end{equation}
Equation (49) translates the perception of the law of motion of the personal wealth $k$. It is a rewriting of equation (6) with the addition of an attention discount factor to each of the perceived variables, where : 
\begin{itemize}
    \item $k_{t+1}$ is the personal wealth at time $t+1$
    \item $\textbf{G}^{k,BR}$ is the transition function of personal wealth under bounded rationality, i.e. the perception of the transition function under bounded rationality
    \item $c_t$ is the consumption at time $t$
    \item $N_{t}$ is work at time $t$
    \item $\bar{r}$ the interest rate at the steady state
    \item $\hat{r}^{BR}$ the perception under bounded rationality of the deviation of the interest rate from the steady state 
    \item $\textbf{X}_t$ the state vector
    \item $\bar{y}$ the income at the steady state 
    \item $\hat{y}^{BR}(N_t,\textbf{X}_{t})$ the perception under bounded rationality of the deviation of the income from the steady state
\end{itemize}

\subsubsection*{Equation 50}

\begin{equation}\tag{50}
    \begin{cases}
        \hat{r}^{BR} = m_{r}\cdot\hat{r}(\textbf{X}_{t}) \\
        \hat{y}^{BR}(N_{t},\textbf{X}_{t}) = m_{y}\cdot\hat{y}(\textbf{X}_{t})+\omega(\textbf{X}_{t})(N_{t}-N_{t}\textbf{X}_{t})
    \end{cases}
\end{equation}

The equation (50) defines the perceived values under bounded rationality of the interest rate and of the income, where : 
\begin{itemize}
    \item $\hat{r}^{BR}$ is the perception of the deviation of the interest rate from the steady state under bounded rationality
    \item $m_{r}\in\left[0,1\right]$ is the attention discount factor for the interest rate
    \item $\hat{r}_{t}(\textbf{X}_{t})$ is the objective value of the deviation of the interest rate from the steady state 
    
    \item $\hat{y}^{BR}(N_{t},\textbf{X}_{t})$ is the perception of the deviation of the personal from the steady state under bounded rationality, which is function of the number of hours worked, and of the macroeconomics state vector at time $t$
    \item $m_{y}\in\left[0,1\right]$ is the attention discount factor for the personal income
    \item $\omega(\textbf{X}_{t})$ is the real wage, which is function of the macroeconomic state vector
    \item $N(\textbf{X}_{t})$ is the aggregate labor supply
\end{itemize}
The perceptions of the laws of motion are thus further refined by differentiating the attention level in function of the economic variable considered. 
Taken together, equations (49) and (50) can be taken as more comple budget constraints, that are to take into account in the maximisation process of the utility function.
Note that both discount factors $m_{r}$ and $m_{y}$ apply to the period $t$, meaning they are contemporaneous discount factor. 

\subsubsection*{Equation 51, Lemma 5 (Term Structure of Attention)}

\begin{equation}\tag{51}
    \begin{cases}
        \mathbb{E}_{t}^{BR}\left[\hat{r}^{BR}(\textbf{X}_{t+k})\right]=m_{r}\bar{m}^{k}\mathbb{E}_{t}\left[\hat{r}(\textbf{X}_{t+k})\right] \\
        \mathbb{E}_{t}^{BR}\left[\hat{y}^{BR}(\textbf{X}_{t+k})\right]=m_{y}\bar{m}^{k}\mathbb{E}_{t}\left[\hat{y}(\textbf{X}_{t+k})\right]
    \end{cases}
\end{equation}

Equation (51) is the application of Lemma 1 (equation (11)) on the perceived values just defined in equations (49) and (50), where : 
\begin{itemize}
    \item $\mathbb{E}_{t}^{BR}$ is the operator of the expected value under bounded rationality, i.e. the behavioral expectation defined in equation (11)
    \item $\hat{y}^{BR}(N_{t},\textbf{X}_{t})$ is the perception of the deviation of the personal from the steady state under bounded rationality, which is function of the number of hours worked, and of the macroeconomics state vector at time $t$
    \item $\hat{r}^{BR}$ is the perception of the deviation of the interest rate from the steady state under bounded rationality
    \item $m_{r}\in\left[0,1\right]$ is the attention discount factor for the interest rate
    \item $m_{y}\in\left[0,1\right]$ is the attention discount factor for the personal income
    \item $\bar{m}$ is the general discount factor applicable for all future variables
\end{itemize}

Equation (51) is the main consequence of the refinment previously done in equations (49) and (50) : not only are the agents generally myopic to the future, but they are also not fully attentive to present variables, and their attention level can vary depending on which one they consider.
It can indeed seem plausible to say that they pay more attention to the income than to the interest rate, even when they have the information about the current period. In this case it would mean that $m_{r}<m{y}$. 
This equation, coupled with the next one, argues that one is not fully rational even when they have acces to the information of their present income and personal Euler equation, because they have this present attention deficiency to income $m_{y}$.

\subsubsection*{Equation 52}

\begin{equation}\tag{52}
        \hat{c}_{t}=\mathbb{E}_{t}\left[\sum_{\tau\geq t}\frac{\bar{m}^{\tau-t}}{R^{\tau-t}}\left(b_{r}m_{r}\hat{r}(\textbf{X}_{\tau})+m_{Y}\frac{\bar{r}}{R}\hat{y}(\textbf{X}_{\tau})\right)\right]
\end{equation}

This equation can be better understood with the precisions : 
\begin{equation*}
    \begin{cases}
        c_{t}=c_{t}^{d}+\hat{c}_{t} \\ 
        c^{d}_{t} = \bar{y} + b_{k}\cdot k_{t} \\
        b_{k}:=\frac{\bar{r}}{R}\cdot\frac{\phi}{\phi+\gamma}
    \end{cases}
\end{equation*}
$$\iff$$
\begin{equation*}
    \hat{c}_{t}=c_{t}-c_{t}^{d}=c_{t}-\bar{y} - \frac{\bar{r}}{R}\cdot\frac{\phi}{\phi+\gamma}\cdot k_{t}
\end{equation*}
To get to the final form of equation (52), we should then plug the expression of (49), and take into account equations (50) and (51).
Globally, equation (52) is the solution for consumption in a model with a structured attention of consumer, that relates to equations (18)-(19) in the baseline model, where : 
\begin{itemize}
    \item $\hat{c}_{t}$ is the value of the deviation from the steady state of consumption
    \item $\tau$ is a time period in the future
    \item $b_{r}:=-\frac{1}{\gamma\cdot R^{2}}$ is the coefficient associated to the interest rate
    \item $m_{Y}=\frac{\phi\cdot m_{y}+\gamma}{\phi+\gamma}$ is the coefficient associated to the income
\end{itemize}

Even though the equation is more complex than in the baseline model, we should retain that the solution of the consumer here is only affected by this variable specific discount dampening. 

\subsubsection*{Equation 52.5 : Discounted Euler Equation, with Term Structure of Attention (Proposition 9)}

Equation 52.5 is not directly defined in the article, but is described as a refinment of equation (24).

\begin{equation*}\tag{52.5}
    x_{t}=M\cdot\mathbb{E}_{t}\left[x_{t+1}\right]-\sigma(\hat{r}_{t}-\hat{r}^{n}_{t})
\end{equation*}

Where : 
\begin{itemize}
    \item $M=\frac{\bar{m}}{(R-r\cdot m_{Y})}\in\left[0,1\right]$ is the "macro", or global parameter of attention
    \item $\sigma:=\frac{m_{r}}{\gamma\cdot R\cdot(R-r\cdot m_{Y})}\in\left[0,\frac{1}{\gamma\cdot R}\right]$ is the coefficient of the effect of the interest rate gap on the output gap
\end{itemize}

Basically, this equation is the result of the refinment of equation (24) with the given attention structure.

\subsubsection*{Equation 53}

\begin{equation}\tag{53}
    \frac{\Delta^{GE}}{\Delta^{\text{direct}}}=R^{\tau+1}
\end{equation}

To understand this equation, we have to add : 

\begin{equation*}
    \begin{cases}
        \Delta^{\text{direct}}:=\frac{\partial \hat{c}_{0}}{\partial \hat{r}_{\tau}}\bigg\rvert_{(y_{t})_{t\geq0 \text{ held constant}}} = -\alpha\cdot \frac{1}{R^{\tau}} \\
        \Delta^{GE}:=\frac{\partial \hat{c}_{0}}{\partial \hat{r}_{\tau}}=-\alpha R 
    \end{cases}
\end{equation*}

Equation (53) describes the effect of a future variation of the interest rate on consumption on the present consumption of \textbf{of the rational agent}, where :

\begin{itemize}
    \item $\Delta^{\text{direct}}$ is the direct effect of the variation of interest rate on present consumption, with no retroaction effect on the income
    \item $\Delta^{GE}$ is the indirect effect of the variation of interest rate on present consumption, with retroaction effect on the income
\end{itemize}

Equation (53) refers to the Keynesian coefficient multiplicator in the rational case. 
Dividing the indirect impact by the direct impact allows to see the magnitude of the general impact of the variation of the interest rate. 

\subsubsection*{Equation 54}

\begin{equation}\tag{54}
    \frac{\Delta^{GE}}{\Delta^{\text{direct}}}=\left(\frac{R}{R-rm_{Y}}\right)^{\tau+1}\in\left[1, R^{\tau+1}\right]
\end{equation}

To understand this equation, we have to add :

\begin{equation*}
    \begin{cases}
        \Delta^{\text{direct}}:=\frac{\partial \hat{c}_{0}}{\partial \hat{r}_{\tau}}\bigg\rvert_{(y_{t})_{t\geq0 \text{ held constant}}} = -\alpha\cdot m_{r}\cdot\bar{m}^{\tau}\frac{1}{R^{\tau}} \\
        \Delta^{GE}:=\frac{\partial \hat{c}_{0}}{\partial \hat{r}_{\tau}}=-\alpha m_{r}\cdot M^{\tau} \frac{R}{R-r\cdot m_{Y}}R 
    \end{cases}
\end{equation*}
    
Where : 
\begin{itemize}
    \item $M= \frac{\bar{m}}{R-r\cdot m_{Y}}$ 
    \item $\Delta^{\text{direct}}$
    \item $\Delta^{\text{GE}}$
\end{itemize}

Equation (54) is the same as equation (53), but withing the beahvioral framework, with a structured attention term. 
We see that within this framework, with a attention dampening for future periods, the magnitude of the interest rate variation is lessened by the attention coefficients. 
This explains why the Keynesian multiplication factor is not as strong as predicted in real life.
Equation (53) only works if consumers believe that other consumers are fully rational, equation (54) does not impose this condition. 

\subsection{Flattening of the Phillips Curve via Imperfect Firm Attention}

\subsubsection*{Equation 55}

\begin{equation}\tag{55}
    \varv^{BR}(q_{it},(\textbf{X}_{\tau})):=\varv^{0}\left(q_{it}-m^{f}_{\pi}\Pi(\textbf{X}_{\tau}),m^{f}_{x}\mu(\textbf{X}_{\tau}),c(\textbf{X}_{\tau})\right)
\end{equation}

Equation (55) introduces specific attention deficiency for firms, where : 
\begin{itemize}
    \item $\varv^{0}(q_{i\tau},\mu_{\tau},c_{\tau}):=\left(e^{q_{i\tau}}-(1-\tau_{f})e^{-\mu_{\tau}}\right)e^{-\varepsilon q_{i\tau}}c_{\tau}$ is the current real profit of the firm, as in equation (14) of the baseline model
    \item $m^{f}_{\pi}$ is the attention deficit to inflation 
    \item $m^{f}_{x}$ is the attention deficit to marginal cost
\end{itemize}

This equation takes the main elements of equation (15), but introduces a variable specific attention deficiency factor for the firms. 
It allows to have a new maximisation program, for firms with bounded rationality, described in the next equation.

\subsubsection*{Equation 56}

\begin{equation}\tag{56}
    \max_{q_{it}}{\mathbb{E}_{t}^{BR}\left[\sum_{\tau=t}^{\infty}\left(\beta\theta\right)^{\tau-t}\frac{c(\textbf{X}_{\tau})^{-\gamma}}{c(\textbf{X}_{\tau})^{-\gamma}}\varv^{BR}(q_{it},\textbf{X}_{\tau})\right]}
\end{equation}

Equation (56) describes the maximisation program of a firm under bounded rationality, it is the same as equation (16), where : 
\begin{itemize}
    \item $\varv^{BR}$ is the perception of the real profit at time $t$, described in equation (55)
\end{itemize}

\subsubsection*{Equation 57}

\begin{equation}\tag{57}
    p^{*}_{t}=p_{t}+(1-\beta\theta)\sum^{\infty}_{k=0} \left(\beta\theta\bar{m}\right)^{k}\mathbb{E}_{t}\left[m^{f}_{\pi}(\pi_{t+1}+...+\pi_{t+k})-m^{f}_{x}\mu_{t+k}\right]
\end{equation}

Equation (57) is the solution to the maximisation program described in equation (56), but with the introduced variable specific attention discount factors, $m^{f}_{\pi}$ and $m^{f}_{x}$.
It is the equivalent of equation (27) in the baseline model. 

\subsubsection*{Equation 57.5 - Proposition 10}

Equation (57.5) is not directly mentioned in the article, but is described as a refined version of equation (29) with different parameters.

\begin{equation}\tag{57.5}
    \pi_{t}=\beta\cdot M^{f}\cdot\mathbb{E}t\left[\pi_{t+1}\right]+\kappa\cdot x_{t}
\end{equation}
$$\iff$$
\begin{equation}\tag{57.5}
    \pi_{t}=\beta\cdot\bar{m}\left(\theta+m^{f}_{\pi}\cdot(1-\theta)\cdot\frac{1-\beta\cdot\theta}{1-\beta\cdot\theta\cdot\bar{m}}\right)\cdot\mathbb{E}t\left[\pi_{t+1}\right]+m_{x}^{f}\cdot\bar{\kappa}\cdot x_{t}
\end{equation}

The two differences are given by : 
\begin{itemize}
    \item $M^{f}=\bar{m}\left(\theta+m^{f}_{\pi}\cdot(1-\theta)\cdot\frac{1-\beta\cdot\theta}{1-\beta\cdot\theta\cdot\bar{m}}\right)\in\left[0,1\right]$ for the general attention factor of the firm
    \item $\kappa=m_{x}^{f}\cdot\bar{\kappa}$, as detailed in the next equation
\end{itemize}

\subsubsection*{Equation 58}

\begin{equation}\tag{58}
    \kappa = m^{f}_{x}\bar{\kappa}
\end{equation}

Equation (58) describes the attention by the firm under bounded rationality to current macroeconomic production, where : 
\begin{itemize}
    \item $m^{f}_{x}$ is the attention deficiency to the output gap
    \item $\bar{\kappa}$ is the slope of the traditional Phillips curve, i.e. the effect coefficient of the output gap on inflation
\end{itemize}

Equation (58) depicts the fact that the firms attention deficiency to the output gap affects the way they perceive their potential profit.

\subsection{Nonconstant Trend Inflation and Neo-Fisherian Paradoxes}

\subsubsection*{Equation 59}

\begin{equation}\tag{59}
    \pi^{d}_{t}=(1-\zeta)\bar{\pi}_{t}+\zeta\bar{\pi}_{t}^{CB}
\end{equation}

Where : 
\begin{itemize}
    \item $\pi^{d}_{t}$ is the "default" inflation value, perceived by the firms
    \item $\bar{\pi}_{t}$ is the moving average of past inflation 
    \item $\bar{\pi}_{t}^{CB}$ is the inflation guidance, i.e. the inflation target declared by the Central Bank
    \item $\zeta\in\left[0,1\right]$ is a weight factor on past inflation, which is not the same as the Total Factor productivity defined in the baseline model
\end{itemize}

Equation (59) describes the fact that firms predict a default value of inflation that is given by a weighted average of what they observed in the past regarding the actual inflation and the central bank policy. 

\subsubsection*{Equation 60}

\begin{equation}\tag{60}
    x_{t}=M\cdot\mathbb{E}_{t}\left[x_{t+1}\right]-\sigma\left(i_{t}-\mathbb{E}_{t}\left[\pi_{t+1}\right]-r^{n}_{t}\right)
\end{equation}

Equation (60) is exactly the same as the IS curve from equation (28).

\subsubsection*{Equation 61}

\begin{equation}\tag{61}
    \hat{\pi}_{t}=\beta\cdot M^{f}\cdot\mathbb{E}_t\left[\hat{\pi}_{t+1}\right]+\kappa\cdot x_{t}
\end{equation}

Equation (61) is obtained by taking into account the nonzero trend inflation in the formulation of inflation. 
This equation is thus the same as (29), but with inflation $\pi_{t}$ replace by $\hat{\pi}_{t}$ the deviation from the default value, where : 
\begin{itemize}
    \item $\pi_{t}=\pi^{d}_{t}+\hat{\pi}_{t}$, with $\pi^{d}_{t}$ the default inflation and $\hat{\pi}_{t}$ the deviation from the default value
\end{itemize}

Together, equations (60) and (61) constitute Proposition 11 of the paper, and describe the behavioral new keynesian model augmented by a nonzero trend inflation.

\subsubsection*{Equation 62, Proposition 12}

\begin{equation}\tag{62}
    \phi_{\pi}+\zeta \frac{(1-\beta M^{f})}{\kappa}\phi_{x}+\zeta\frac{(1-\beta M^{f})(1-M)}{\kappa \sigma}>1
\end{equation}

Equation (62) describes the Equilibrium Determinacy of the model with Behavioral Agents.
It is the condition for the model to be determinate in this refined framework, i.e. the equivalent of equation (34) in the baseline model.
All the terms have been defined previously. 


\end{document}