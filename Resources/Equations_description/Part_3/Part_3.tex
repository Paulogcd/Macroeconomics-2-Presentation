\documentclass{article}
\usepackage{amsmath}
\usepackage{amssymb}
\usepackage{bm}
\usepackage{hyperref}
\usepackage{csquotes}

\DeclareSymbolFont{matha}{OML}{txmi}{m}{it}% txfonts
\DeclareMathSymbol{\varv}{\mathord}{matha}{118}

\title{Macroeconomics 2 Presentation \\ Part III equations}
\author{Gugelmo Cavalheiro Dias Paulo \\ \and Mitash Nayanika \\ \and Wang Shang}
\date{\today}

\begin{document}

\maketitle

\section{Behavioral Enrichments to the Model}

\subsection{Term Structure of Consumer Attention}

\subsubsection*{Equation 49}

\begin{equation} \tag{49}
    \begin{split}
        k_{t+1}= &\  \textbf{G}^{k,BR}(c_{t},N_{t},k_{t},\textbf{X}_{t}) \\
        & := (1+\bar{r}+\hat{r}^{BR}(\textbf{X}_t))(k_{t}+\bar{y}+\hat{y}^{BR}(N_{t},\textbf{X}_t)-c_{t})
    \end{split}
\end{equation}
Equation (49) translates the perception of the law of motion of the personal wealth $k$. It is a rewriting of equation (6) with the addition of an attention discount factor to each of the perceived variables, where : 
\begin{itemize}
    \item $k_{t+1}$ is the personal wealth at time $t+1$
    \item $\textbf{G}^{k,BR}$ is the transition function of personal wealth under bounded rationality, i.e. the perception of the transition function under bounded rationality
    \item $c_t$ is the consumption at time $t$
    \item $N_{t}$ is work at time $t$
    \item $\bar{r}$ the interest rate at the steady state
    \item $\hat{r}^{BR}$ the perception under bounded rationality of the deviation of the interest rate from the steady state 
    \item $\textbf{X}_t$ the state vector
    \item $\bar{y}$ the income at the steady state 
    \item $\hat{y}^{BR}(N_t,\textbf{X}_{t})$ the perception under bounded rationality of the deviation of the income from the steady state
\end{itemize}

\subsubsection*{Equation 50}

\begin{equation}\tag{50}
    \begin{cases}
        \hat{r}^{BR} = m_{r}\cdot\hat{r}(\textbf{X}_{t}) \\
        \hat{y}^{BR}(N_{t},\textbf{X}_{t}) = m_{y}\cdot\hat{y}(\textbf{X}_{t})+\omega(\textbf{X}_{t})(N_{t}-N_{t}(\textbf{X}_{t}))
    \end{cases}
\end{equation}

The equation (50) defines the perceived values under bounded rationality of the interest rate and of the income, where : 
\begin{itemize}
    \item $\hat{r}^{BR}$ is the perception of the deviation of the interest rate from the steady state under bounded rationality
    \item $m_{r}\in\left[0,1\right]$ is the attention discount factor for the interest rate
    \item $\hat{r}_{t}(\textbf{X}_{t})$ is the objective value of the deviation of the interest rate from the steady state 
    
    \item $\hat{y}^{BR}(N_{t},\textbf{X}_{t})$ is the perception of the deviation of the personal from the steady state under bounded rationality, which is function of the number of hours worked, and of the macroeconomics state vector at time $t$
    \item $\hat{y}^{BR}(\textbf{X}_{t})=\hat{y}^{BR}(N(\textbf{X}_{t}),\textbf{X}_{t})=\hat{y}(\textbf{X}_{t})$ is the perceived deviation value of the aggregate income
    \item $m_{y}\in\left[0,1\right]$ is the attention discount factor for the personal income
    \item $\omega(\textbf{X}_{t})$ is the real wage
    \item $N(\textbf{X}_{t})$ is the aggregate labor supply
\end{itemize}
The perceptions of the laws of motion are thus further refined by differentiating the attention level in function of the economic variable considered. 
Taken together, equations (49) and (50) can be taken as more comple budget constraints, that are to take into account in the maximisation process of the utility function.
Note that both discount factors $m_{r}$ and $m_{y}$ apply to the period $t$, meaning they are contemporaneous discount factor. 

\subsubsection*{Equation 51, Lemma 5 (Term Structure of Attention)}

\begin{equation}\tag{51}
    \begin{cases}
        \mathbb{E}_{t}^{BR}\left[\hat{r}^{BR}(\textbf{X}_{t+k})\right]=m_{r}\cdot\bar{m}^{k}\cdot\mathbb{E}_{t}\left[\hat{r}(\textbf{X}_{t+k})\right] \\
        \mathbb{E}_{t}^{BR}\left[\hat{y}^{BR}(\textbf{X}_{t+k})\right]=m_{y}\cdot\bar{m}^{k}\cdot\mathbb{E}_{t}\left[\hat{y}(\textbf{X}_{t+k})\right]
    \end{cases}
\end{equation}

Equation (51) is the application of Lemma 1 (equation (11)) on the perceived values just defined in equations (49) and (50), where : 
\begin{itemize}
    \item $\mathbb{E}_{t}^{BR}$ is the operator of the expected value under bounded rationality, i.e. the behavioral expectation defined in equation (11)
    \item $\hat{y}^{BR}(N_{t},\textbf{X}_{t})$ is the perception of the deviation of the personal from the steady state under bounded rationality, which is function of the number of hours worked, and of the macroeconomics state vector at time $t$
    \item $\hat{r}^{BR}$ is the perception of the deviation of the interest rate from the steady state under bounded rationality
    \item $m_{r}\in\left[0,1\right]$ is the attention discount factor for the interest rate
    \item $m_{y}\in\left[0,1\right]$ is the attention discount factor for the personal income
    \item $\bar{m}$ is the general discount factor applicable for all future variables
\end{itemize}

Equation (51) is the main consequence of the refinment previously done in equations (49) and (50) : not only are the agents generally myopic to the future, but they are also not fully attentive to present variables, and their attention level can vary depending on which one they consider.
It can indeed seem plausible to say that they pay more attention to the income than to the interest rate, even when they have the information about the current period. In this case it would mean that $m_{r}<m{y}$. 
This equation, coupled with the next one, argues that one is not fully rational even when they have acces to the information of their present income and personal Euler equation, because they have this present attention deficiency to income $m_{y}$.

\subsubsection*{Equation 52}

\begin{equation}\tag{52}
        \hat{c}_{t}=\mathbb{E}_{t}\left[\sum_{\tau\geq t}\frac{\bar{m}^{\tau-t}}{R^{\tau-t}}\left(b_{r}m_{r}\hat{r}(\textbf{X}_{\tau})+m_{Y}\frac{\bar{r}}{R}\hat{y}(\textbf{X}_{\tau})\right)\right]
\end{equation}

This equation can be better understood with the precisions : 
\begin{equation*}
    \begin{cases}
        c_{t}=c_{t}^{d}+\hat{c}_{t} \\ 
        c^{d}_{t} = \bar{y} + b_{k}\cdot k_{t} \\
        b_{k}:=\frac{\bar{r}}{R}\cdot\frac{\phi}{\phi+\gamma}
    \end{cases}
\end{equation*}
$$\iff$$
\begin{equation*}
    \hat{c}_{t}=c_{t}-c_{t}^{d}=c_{t}-\bar{y} - \frac{\bar{r}}{R}\cdot\frac{\phi}{\phi+\gamma}\cdot k_{t}
\end{equation*}
To get to the final form of equation (52), we should then plug the expression of (49), and take into account equations (50) and (51).
Globally, equation (52) is the solution for consumption in a model with a structured attention of consumer, that relates to equations (18)-(19) in the baseline model, where : 
\begin{itemize}
    \item $\hat{c}_{t}$ is the value of the deviation from the steady state of consumption
    \item $\tau$ is a time period in the future
    \item $b_{r}:=-\frac{1}{\gamma\cdot R^{2}}$ is the coefficient associated to the interest rate
    \item $m_{Y}=\frac{\phi\cdot m_{y}+\gamma}{\phi+\gamma}$ is the coefficient associated to the income
\end{itemize}

Even though the equation is more complex than in the baseline model, we should retain that the solution of the consumer here is only affected by this variable specific discount dampening. 

\subsubsection*{Equation 52.5 : Discounted Euler Equation, with Term Structure of Attention (Proposition 9)}

Equation 52.5 is not directly defined in the article, but is described as a refinment of equation (24).

\begin{equation*}\tag{52.5}
    x_{t}=M\cdot\mathbb{E}_{t}\left[x_{t+1}\right]-\sigma(\hat{r}_{t}-\hat{r}^{n}_{t})
\end{equation*}

Where : 
\begin{itemize}
    \item $M=\frac{\bar{m}}{(R-r\cdot m_{Y})}\in\left[0,1\right]$ is the "macro", or global parameter of attention
    \item $\sigma:=\frac{m_{r}}{\gamma\cdot R\cdot(R-r\cdot m_{Y})}\in\left[0,\frac{1}{\gamma\cdot R}\right]$ is the coefficient of the effect of the interest rate gap on the output gap
\end{itemize}

Basically, this equation is the result of the refinment of equation (24) with the given attention structure.

\subsubsection*{Equation 53}

\begin{equation}\tag{53}
    \frac{\Delta^{GE}}{\Delta^{\text{direct}}}=R^{\tau+1}
\end{equation}

To understand this equation, we have to add : 

\begin{equation*}
    \begin{cases}
        \Delta^{\text{direct}}:=\frac{\partial \hat{c}_{0}}{\partial \hat{r}_{\tau}}\bigg\rvert_{(y_{t})_{t\geq0 \text{ held constant}}} = -\alpha\cdot \frac{1}{R^{\tau}} \\
        \Delta^{GE}:=\frac{\partial \hat{c}_{0}}{\partial \hat{r}_{\tau}}=-\alpha R 
    \end{cases}
\end{equation*}

Equation (53) describes the effect of a future variation of the interest rate on consumption on the present consumption of \textbf{of the rational agent}, where :

\begin{itemize}
    \item $\Delta^{\text{direct}}$ is the direct effect of the variation of interest rate on present consumption, with no retroaction effect on the income
    \item $\Delta^{GE}$ is the indirect effect of the variation of interest rate on present consumption, with retroaction effect on the income
\end{itemize}

Equation (53) refers to the Keynesian coefficient multiplicator in the rational case. 
Dividing the indirect impact by the direct impact allows to see the magnitude of the general impact of the variation of the interest rate. 

\subsubsection*{Equation 54}

\begin{equation}\tag{54}
    \frac{\Delta^{GE}}{\Delta^{\text{direct}}}=\left(\frac{R}{R-rm_{Y}}\right)^{\tau+1}\in\left[1, R^{\tau+1}\right]
\end{equation}

To understand this equation, we have to add :

\begin{equation*}
    \begin{cases}
        \Delta^{\text{direct}}:=\frac{\partial \hat{c}_{0}}{\partial \hat{r}_{\tau}}\bigg\rvert_{(y_{t})_{t\geq0 \text{ held constant}}} = -\alpha\cdot m_{r}\cdot\bar{m}^{\tau}\frac{1}{R^{\tau}} \\
        \Delta^{GE}:=\frac{\partial \hat{c}_{0}}{\partial \hat{r}_{\tau}}=-\alpha m_{r}\cdot M^{\tau} \frac{R}{R-r\cdot m_{Y}}R 
    \end{cases}
\end{equation*}
    
Where : 
\begin{itemize}
    \item $M= \frac{\bar{m}}{R-r\cdot m_{Y}}$ 
    \item $\Delta^{\text{direct}}$
    \item $\Delta^{\text{GE}}$
\end{itemize}

Equation (54) is the same as equation (53), but withing the beahvioral framework, with a structured attention term. 
We see that within this framework, with a attention dampening for future periods, the magnitude of the interest rate variation is lessened by the attention coefficients. 
This explains why the Keynesian multiplication factor is not as strong as predicted in real life.
Equation (53) only works if consumers believe that other consumers are fully rational, equation (54) does not impose this condition. 

\subsection{Flattening of the Phillips Curve via Imperfect Firm Attention}

\subsubsection*{Equation 55}

\begin{equation}\tag{55}
    \varv^{BR}(q_{it},(\textbf{X}_{\tau})):=\varv^{0}\left(q_{it}-m^{f}_{\pi}\Pi(\textbf{X}_{\tau}),m^{f}_{x}\mu(\textbf{X}_{\tau}),c(\textbf{X}_{\tau})\right)
\end{equation}

Equation (55) introduces specific attention deficiency for firms, where : 
\begin{itemize}
    \item $\varv^{0}(q_{i\tau},\mu_{\tau},c_{\tau}):=\left(e^{q_{i\tau}}-(1-\tau_{f})e^{-\mu_{\tau}}\right)e^{-\varepsilon q_{i\tau}}c_{\tau}$ is the current real profit of the firm, as in equation (14) of the baseline model
    \item $m^{f}_{\pi}$ is the attention deficit to inflation 
    \item $m^{f}_{x}$ is the attention deficit to marginal cost
\end{itemize}

This equation takes the main elements of equation (15), but introduces a variable specific attention deficiency factor for the firms. 
It allows to have a new maximisation program, for firms with bounded rationality, described in the next equation.

\subsubsection*{Equation 56}

\begin{equation}\tag{56}
    \max_{q_{it}}{\mathbb{E}_{t}^{BR}\left[\sum_{\tau=t}^{\infty}\left(\beta\theta\right)^{\tau-t}\frac{c(\textbf{X}_{\tau})^{-\gamma}}{c(\textbf{X}_{t})^{-\gamma}}\varv^{BR}(q_{it},\textbf{X}_{\tau})\right]}
\end{equation}

Equation (56) describes the maximisation program of a firm under bounded rationality, it is the same as equation (16), where : 
\begin{itemize}
    \item $\varv^{BR}$ is the perception of the real profit at time $t$, described in equation (55)
\end{itemize}

\subsubsection*{Equation 57}

\begin{equation}\tag{57}
    p^{*}_{t}=p_{t}+(1-\beta\theta)\sum^{\infty}_{k=0} \left(\beta\theta\bar{m}\right)^{k}\mathbb{E}_{t}\left[m^{f}_{\pi}(\pi_{t+1}+...+\pi_{t+k})-m^{f}_{x}\mu_{t+k}\right]
\end{equation}

Equation (57) is the solution to the maximisation program described in equation (56), but with the introduced variable specific attention discount factors, $m^{f}_{\pi}$ and $m^{f}_{x}$.
It is the equivalent of equation (27) in the baseline model. 

\subsubsection*{Equation 57.5 - Proposition 10}

Equation (57.5) is not directly mentioned in the article, but is described as a refined version of equation (29) with different parameters.

\begin{equation}\tag{57.5}
    \pi_{t}=\beta\cdot M^{f}\cdot\mathbb{E}t\left[\pi_{t+1}\right]+\kappa\cdot x_{t}
\end{equation}
$$\iff$$
\begin{equation}\tag{57.5}
    \pi_{t}=\beta\cdot\bar{m}\left(\theta+m^{f}_{\pi}\cdot(1-\theta)\cdot\frac{1-\beta\cdot\theta}{1-\beta\cdot\theta\cdot\bar{m}}\right)\cdot\mathbb{E}t\left[\pi_{t+1}\right]+m_{x}^{f}\cdot\bar{\kappa}\cdot x_{t}
\end{equation}

The two differences are given by : 
\begin{itemize}
    \item $M^{f}=\bar{m}\left(\theta+m^{f}_{\pi}\cdot(1-\theta)\cdot\frac{1-\beta\cdot\theta}{1-\beta\cdot\theta\cdot\bar{m}}\right)\in\left[0,1\right]$ for the general attention factor of the firm
    \item $\kappa=m_{x}^{f}\cdot\bar{\kappa}$, as detailed in the next equation
\end{itemize}

\subsubsection*{Equation 58}

\begin{equation}\tag{58}
    \kappa = m^{f}_{x}\bar{\kappa}
\end{equation}

Equation (58) describes the attention by the firm under bounded rationality to current macroeconomic production, where : 
\begin{itemize}
    \item $m^{f}_{x}$ is the attention deficiency to the output gap
    \item $\bar{\kappa}$ is the slope of the traditional Phillips curve, i.e. the effect coefficient of the output gap on inflation
\end{itemize}

Equation (58) depicts the fact that the firms attention deficiency to the output gap affects the way they perceive their potential profit.

\subsection{Nonconstant Trend Inflation and Neo-Fisherian Paradoxes}

\subsubsection*{Equation 59}

\begin{equation}\tag{59}
    \pi^{d}_{t}=(1-\zeta)\bar{\pi}_{t}+\zeta\bar{\pi}_{t}^{CB}
\end{equation}

Where : 
\begin{itemize}
    \item $\pi^{d}_{t}$ is the "default" inflation value, perceived by the firms
    \item $\bar{\pi}_{t}$ is the moving average of past inflation 
    \item $\bar{\pi}_{t}^{CB}$ is the inflation guidance, i.e. the inflation target declared by the Central Bank
    \item $\zeta\in\left[0,1\right]$ is a weight factor on past inflation, which is not the same as the Total Factor productivity defined in the baseline model
\end{itemize}

Equation (59) describes the fact that firms predict a default value of inflation that is given by a weighted average of what they observed in the past regarding the actual inflation and the central bank policy. 

\subsubsection*{Equation 60}

\begin{equation}\tag{60}
    x_{t}=M\cdot\mathbb{E}_{t}\left[x_{t+1}\right]-\sigma\left(i_{t}-\mathbb{E}_{t}\left[\pi_{t+1}\right]-r^{n}_{t}\right)
\end{equation}

Equation (60) is exactly the same as the IS curve from equation (28).

\subsubsection*{Equation 61}

\begin{equation}\tag{61}
    \hat{\pi}_{t}=\beta\cdot M^{f}\cdot\mathbb{E}_t\left[\hat{\pi}_{t+1}\right]+\kappa\cdot x_{t}
\end{equation}

Equation (61) is obtained by taking into account the nonzero trend inflation in the formulation of inflation. 
This equation is thus the same as (29), but with inflation $\pi_{t}$ replace by $\hat{\pi}_{t}$ the deviation from the default value, where : 
\begin{itemize}
    \item $\pi_{t}=\pi^{d}_{t}+\hat{\pi}_{t}$, with $\pi^{d}_{t}$ the default inflation and $\hat{\pi}_{t}$ the deviation from the default value
\end{itemize}

Together, equations (60) and (61) constitute Proposition 11 of the paper, and describe the behavioral new keynesian model augmented by a nonzero trend inflation.

\subsubsection*{Equation 62, Proposition 12}

\begin{equation}\tag{62}
    \phi_{\pi}+\zeta \frac{(1-\beta M^{f})}{\kappa}\phi_{x}+\zeta\frac{(1-\beta M^{f})(1-M)}{\kappa \sigma}>1
\end{equation}

Equation (62) describes the Equilibrium Determinacy of the model with Behavioral Agents.
It is the condition for the model to be determinate in this refined framework, i.e. the equivalent of equation (34) in the baseline model.
All the terms have been defined previously. 

\end{document}