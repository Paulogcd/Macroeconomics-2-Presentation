\documentclass{article}
\usepackage{amsmath}
\usepackage{amssymb}
\usepackage{bm}
\DeclareSymbolFont{matha}{OML}{txmi}{m}{it}% txfonts
\DeclareMathSymbol{\varv}{\mathord}{matha}{118}

\title{Macroeconomics 2 Presentation \\ Part III equations}
\author{Gugelmo Cavalheiro Dias Paulo \\ \and Mitash Nayanika \\ \and Wang Shang}
\date{\today}


\begin{document}

\section{A Behavioral Model}
Let's ignore the first two equations, since they are the same as (28) and (29), that will be explained later.

\subsection{Introduction}

\subsubsection*{Equation 1}
% equation 1%
\begin{equation}
    x_{t}=M\cdot\mathbb{E}_{t}\left[ x_{t+1} -\sigma (i_{t}-\mathbb{E}_t\left[\pi_{t+1}\right]-r^{n}_{t})\right]
\end{equation}

\subsubsection*{Equation 2}
% equation 2%
\begin{equation}
    \pi_{t}=\beta\cdot M^{f} \mathbb{E}_t\left[\pi_{t+1}\right]+\kappa\cdot x_{t}
\end{equation}

\subsection{Basic Setup and the Household’s Problem}

\subsubsection*{Equation 3}
% equation 3%
\begin{equation}
    U = \mathbb{E}_t \left[ \sum_{t=0}^{\infty} \beta^{t}\left(\frac{c_{t}^{1-\gamma}-1}{1-\gamma} - \frac{N_{t}^{1+\phi}}{1+\phi}\right)\right]
\end{equation}

Equation (3) is just the flow utility of the Household, with : 
\begin{itemize}
    \item $\beta$ the discount factor
    \item $c_{t}$ the consumption of the houshold at time $t$
    \item $N_{t}$ the work of the household at time $t$
    \item $\gamma$ determines the concavity of the utility function with respect to the consumption, i.e. the importance of consumption in the utility function
    \item $\phi$ determines the concavity of the utility function with respect to work, i.e. the importance of work in the utility function
\end{itemize}

%There is some computation between (3) and (4) !!!%

\subsubsection*{Equation 4}
% equation 4%
\begin{equation}
    k_{t+1}=(1+r_t)(k_{t}-c_{t}+y_{t})
\end{equation}

Equation (4) is the law of motion of the real financial wealth of the household, where : 
\begin{itemize}
    \item $k_t$ is the real financial welath of the household at time $t$
    \item $r_t$ is the real interest rate
    \item $w_t$ is the real wage
    \item $y_t$ is the agent's real income, defined as $y_t=w_t\cdot N_{t}+y_{t}^{f}$, with $y_{t}^{f}$ the profit income (or the income from firms) at time $t$
\end{itemize}

\subsubsection*{Equation 5}

% equation 5%
\begin{equation}
    \bm{X}_{t+1}=\bm{G}^{\bm{X}}\left(\bm{X}_{t},\bm{\epsilon}_{t+1}\right)
\end{equation}

Equation (5) describes the evolution of macroeconomic variables, where : 
\begin{itemize}
    \item $\bm{X}_t$ is the state vector, including several macroeconomic variables of time $t$, like $\zeta_t$ the aggregate TFP, and the announced actions in monetary and fiscal policy
    \item $\bm{G}^X$ the equilibrium transition function, i.e. the function that gives the macroeconomic variables at time $t+1$ from the macroeconomic variables at the previous period
    \item $\epsilon_{t}$ is the innovation in the economy at time $t$, with $\mathbb{E}_{t}\left[\epsilon_{t+1}\right]=0$, that depends on the equilibrium policies of the agent and of the government
\end{itemize}

\subsubsection*{Equation 6}

% equation 6%
\begin{equation}
    k_{t+1}=G^{k}(c_{t},N_{t}, k_{t}, \bm{X}_t):= (1+\bar{r}+\hat{r}(\bm{X}_{t}))(k_{t}+\bar{y}+\hat{y}(N_{t},\bm{X}_t)-c_{t})
\end{equation}

Equation (6) is the application of the consideration of a set of macroeconomic variables on the law of motion of real financial wealth $k_{t}$, where : 
\begin{itemize}
    \item $\bar{r}$ is the steady state value of the real interest rate, that does not depend on time
    \item $\hat{r}(\bm{X}_t)$ is the value of the deviation from the steady state of the real interest rate, that depends on the state vector $\bm{X}_{t}$ at time $t$
    \item $\bar{y}$ is the steady state value of the agent's real income, that does not depend on time
    \item $\hat{y}(N_{t},\bm{X}_{t})$ is the deviation from the steady state of the agent's real income, that depends on the number of hours worked at time $t$ and on the state vector at time $t$
    \item $c_{t}$ is the aggregate consumption level at time $t$ of the agent
\end{itemize}

\subsubsection*{Equation 7}

% equation 7%
\begin{equation}
    \bm{X}_{t+1}=\bm{\Gamma}\bm{X}_{t}+\bm{epsilon}_{t+1}
\end{equation}

Equation (7) describes the linear version of the equilibrium transition function, it is the linearization of the law of motion, where :
\begin{itemize}
    \item $\Gamma$ is a squared matrix that multiplies the state vector
    \item $\bm{X}_t$ is the state vector at time t
    \item $\epsilon_{t}$ is the innovation shock
\end{itemize}

\subsubsection*{Equation 8 (Assumption 1)}

% equation 8%
\begin{equation}
    \bm{X}_{t+1}=\bar{m}\cdot\bm{G}^{\bm{X}}(\bm{X}_{t},\bm{\epsilon}_{t+1})
\end{equation}

Equation (8) describes the Cognitive Discounting of the State Vector, i.e. the perception by behavioral agents of the law of motion of the macroeconomic variables, where : 
\begin{itemize}
    \item $\bar{m}\in\left[0n1\right]$ is the coginitive discount factor measuring the attention to the future
\end{itemize}

\subsubsection*{Equation 9}

% equation 9%
\begin{equation}
    \bm{X}_{t+1}=\bar{m}(\bm{\Gamma}\bm{X}_{t}+\bm{\epsilon}_{t+1})
\end{equation}

Equation (9) is just the linearized version of the perception by behavioral agents of the law of motion of the state vector. 

\subsubsection*{Equation 10}

\begin{equation}
    \mathbb{E}_{t}^{BR}\left[\bm{X}_{t+k}\right]=\bar{m}^{k}\mathbb{E}_{t}\left[\bm{X}_{t+k}\right]
\end{equation}

Equation (10) defines the expectation of behavioral agents in function of the rational perception of the law of motion of the state vector, where : 
\begin{itemize}
    \item $k\geq 0$ a time period in discrete context
    \item $\mathbb{E}_{t}^{BR}\left[\bm{X}_{t+k}\right]$ is the expected value of the state vector at time $t+k$ by behavioral agents (or subjective/behavioral expectation operator)
    \item $\bar{m}^{k}$ is the cognitive discounting effect at period $t+k$
    \item $\mathbb{E}_{t}\left[\bm{X}_{t+k}\right]$ is the rational expectation of the state vector at time $t+k$
\end{itemize}

\subsubsection*{Equation 11 (Lemma 1)}

\begin{equation}
    \mathbb{E}_{t}^{BR}\left[z\left(\bm{X}_{t+k}\right)\right]=\bar{m}^{k}\mathbb{E}_{t}\left[z\left(\bm{X}_{t+k}\right)\right]
\end{equation}

Equation (11) defines in the general case the behavioral expectation operator, for any function of the state vector, where : 
\begin{itemize}
    \item $k\geq 0$ a time period in discrete context
    \item $z(\cdot)$ is a function, such that $z(0)=0$
    \item $\mathbb{E}_{t}^{BR}\left[z\left(\bm{X}_{t+k}\right)\right]$ is the expected value of the image of the state vector by the function $z(\cdot)$ at time $t+k$ by behavioral agents
    \item $\bar{m}^{k}$ is the cognitive discounting effect at period $t+k$
    \item $\mathbb{E}_{t}\left[z\left(\bm{X}_{t+k}\right)\right]$ is the rational expectation of the image of the state vector by the function $z(\cdot)$ at time $t+k$
\end{itemize}

\subsubsection*{Equation 12}

\begin{equation}
    \mathbb{E}_{t}^{BR}\left[\bar{r}+\hat{r}\left(\bm{X}_{t+k}\right)\right]=\bar{r}+\bar{m}^{k}\mathbb{E}_{t}\left[\hat{r}(\bm{X}_{t+k})\right]
\end{equation}

Equation (12) is an example of the Lemma 1 applied to the interest rate, where :
\begin{itemize}
    \item $k\geq 0$ a time period in discrete context
    \item $\bar{r}$ the steady state level of the real interest rate, that does not depend on time,  
    \item $\hat{r}(\bm{X}_{t+k})$ is the equilibrium transition function defining the value of the deviation from the steady state of the real interest rate in function of the state vector at time $t+k$
    \item $\bar{r}+\hat{r}(\bm{X}_{t})=r_{t}(\bm{X}_{t})$ is the value of the real interest rate at time $t$
    \item $\mathbb{E}_{t}^{BR}\left[\bar{r}+\hat{r}(\bm{X}_{t+k})\right]$ is the expected value of the real interest at time $t+k$ by behavioral agents
    \item $\mathbb{E}_{t}\left[\hat{r}(\bm{X}_{t+k})\right]$ is the rational expectation of value of the deviation of the real itnerest rate from the steady state at time $t+k$
\end{itemize}

\subsection{The Firm's problem}

\subsubsection*{Equation 13}
\begin{equation}
    P_{t}=\left(\int_{0}^{1}P_{it}^{1-\varepsilon}\,di\right)^{\frac{1}{1-\varepsilon}}
\end{equation}

Equation (13) describes the aggregate price level, where : 
\begin{itemize}
    \item $P_{t}$ is the aggregate price level of the economy at time $t$
    \item $i\in\left[0,1\right]$ is the firm index
    \item $\varepsilon$ is the elasticity of subsititution between goods
\end{itemize}

\subsubsection*{Equation 14}
\begin{equation}
    \varv^{0}(q_{i\tau},\mu_{\tau},c_{\tau}):=\left(e^{q_{i\tau}}-(1-\tau_{f})e^{-\mu_{\tau}}\right)e^{-\varepsilon q_{i\tau}}c_{\tau}
\end{equation}

Equation (14) describes the profit of the firm before the lump sum tax of the government, where : 
\begin{itemize}
    \item $\varv$ is the real profit of the firm
    \item $q_{i\tau}=\text{ln}\left(\frac{P_{i\tau}}{P_{\tau}}\right)=p_{i\tau}-p_{\tau}$ is the real log price at time $\tau$
    \item $\tau_f=\frac{1}{\varepsilon}$ it the corrective wage subsidy from the government, funded by the lump sum tax
    \item $\mu_{\tau}=\zeta_{t}-\text{ln}(\omega_{t})$ is the labor wedge, which is zero at efficiency
    \item $\varepsilon$ is the elasticity of subsititution between goods
    \item $c_{\tau}$ is the aggregate level of consumption
\end{itemize}

\subsubsection*{Equation 15}
\begin{equation}
    \varv\left(q_{it},\bm{X}_{\tau}\right):=\varv^{0}\left(q_{it}-\Pi(\bm{X}_{\tau}),\mu(\bm{X}_{\tau}),c(\bm{X}_{\tau})\right)
\end{equation}

Equation 15 describes the flow profit of the firm in function of the real log price and of the extended macro state vector, where : 
\begin{itemize}
    \item $q_{it}=\text{ln}\left(\frac{P_{it}}{P_{t}}\right)=p_{it}-p_{t}$ is the real log price
    \item $\bm{X}_{\tau}=(\bm{X}^{\mathcal{M}}_{\tau}, \Pi_{\tau})$ is the extended macro state vector, with $\bm{X}^{\mathcal{M}_{\tau}}$ the vector of macro variables, including $\zeta_{\tau}$ and possible announcements about future policy
    \item $\Pi\left(\bm{X}_{\tau}\right):=p_{\tau}-p_{t}=\pi_{t+1}+...+\pi_{\tau}$ is the inflation between times $t$ and $\tau$
    \item $q_{it}-\Pi(\bm{X}_{\tau})=q_{i\tau}$ is the real price of the firm if they didn't change its price between $t$ and $\tau$
    \item $\mu\left(\bm{X}_{\tau}\right)$ is the labor wedge in function of the extended state vector at time $t$
    \item $c(\bm{X}_{\tau})$ is the aggregate consumption level in function of the extended state vector at time $t$
\end{itemize}

\subsubsection*{Equation 16}

\begin{equation}
    \max_{q_{it}}{\mathbb{E}_{t}\left[ \sum_{\tau=t}^{\infty}\left(\beta\theta\right)^{\tau-t}\frac{c\left(\bm{X}_{\tau}^{-\gamma}\right)}{c\left(\bm{X}_{t}^{-\gamma}\right)}\varv\left(q_{it,\bm{X}_{\tau}}\right)\right] }
\end{equation}

Equation (16) describes the maximisation program of the firm given that they have a Calvo-like probability of $\theta$ of being able to change their price at each period, where : 

\begin{itemize}
    \item $t$ is the initial period 
    \item $\tau$ is the time period index  
    \item $q_{it}$ is the real log price of the firm at time $t$
    \item $\beta$ is the discount factor
    \item $\theta$ is the Calvo like probability that the firm can change its price at any period
    \item $\frac{c\left(\bm{X}_{\tau}^{-\gamma}\right)}{c\left(\bm{X}_{t}^{-\gamma}\right)}$ is the adjustment in the stochastic discount factor between times $t$ and $\tau$
\end{itemize}

\subsubsection*{Equation 17}
\begin{equation}
    \max_{q_{it}}{\mathbb{E}_{t}^{BR} \left[\sum_{\tau=t}^{\infty}(\beta\theta)^{\tau-t}\frac{c\left(\bm{X}_{\tau}^{-\gamma}\right)}{c\left(\bm{X}_{t}^{-\gamma}\right)} \varv\left(q_{it,\bm{X}_{\tau}}\right) \right]}
\end{equation}

Equation 17 describes the maximisation program of the behavioral firm, i.e. such that it is maximisation the behavioral expectation operator of the flow profit, where : 
\begin{itemize}
    \item $\mathbb{E}_{t}^{BR}$ is the behavioral/subjective expected value operator 
    \item $t$ is the initial period 
    \item $\tau$ is the time period index  
    \item $q_{it}$ is the real log price of the firm at time $t$
    \item $\beta$ is the discount factor
    \item $\theta$ is the Calvo like probability that the firm can change its price at any period
    \item $\frac{c\left(\bm{X}_{\tau}^{-\gamma}\right)}{c\left(\bm{X}_{t}^{-\gamma}\right)}$ is the adjustment in the stochastic discount factor between times $t$ and $\tau$
\end{itemize}

\subsection{Model solution}

\subsubsection*{Equation 18}
\begin{equation}
    \hat{c}_{t}=\mathbb{E}_{t}\left[\hat{c}_{t+1}-\frac{1}{\gamma R}\hat{r}_{t}\right]
\end{equation}

Equation (18) is the linearized version of the Euler equation obtained from the presented model. It is also called the investment-savings (IS) curve, where : 
\begin{itemize}
    \item $\hat{c}_{t}$ is the value of the deviation from the steady state of the aggregate consumption at time $t$
    \item $\mathbb{E}_{t}\left[\hat{c}_{t+1}\right]$ is the rational expectation of the value of the deviation from the steady state of the aggregate consumption at time $t+1$
    \item $\gamma$ is the factor of the importance of consumption
    \item $R:=1+\bar{r}$ is defined from the real intereste rate at the steady state (cf. page 7 of the article)
\end{itemize}

\subsubsection*{Equation 19}
\begin{equation}
    \hat{c}_{t}=M\cdot\mathbb{E}_{t}\left[\hat{c}_{t+1}-\sigma\hat{r}_{t}\right]
\end{equation}

Equation (19) is the application of Lemma 1 (equation (11)) on the previous Euler equation, i.e. a cognitively discounted aggregate Euler equation, where : 
\begin{itemize}
    \item $M$ is the macro parameter of attention, such that $M=\bar{m}$ here
    \item $\sigma=\frac{1}{\gamma R}$
\end{itemize}

\subsubsection*{Equation 20}
\begin{equation}
    N^{\phi}_{t}=\omega_{t}c_{t}^{\gamma}
\end{equation}

Equation (20) the result of the static First Order Condition for labor supply, where : 
\begin{itemize}
    \item $N_{t}$ is the quantity of labor provided at time $t$
    \item $\omega_{t}$ is the real wage at time $t$
    \item $c_{t}$ is the aggregate quantity of consumption at time $t$
    \item $\gamma$ is the consumption importance in the utility
\end{itemize}

\subsubsection*{Equation 21}
\begin{equation}
    \hat{c}_{t}^{n}=\frac{1+\phi}{\gamma+\phi}\zeta_{t}
\end{equation}

Equation (21) ..., where : 
\begin{itemize}
    \item 
\end{itemize}

\subsubsection*{Equation 22}
\begin{equation}
    \hat{c}^{n}_{t} = M\cdot\mathbb{E}_{t}\left[\hat{c}^{n}_{t+1}\right]-\sigma\hat{r}^{n}_{t}
\end{equation}

Equation (22) ..., where : 
\begin{itemize}
    \item 
\end{itemize}

\subsubsection*{Equation 23}
\begin{equation}
    r^{n0}_{t}=\bar{r}+\frac{1+\phi}{\sigma(\gamma+\phi)}\left(M\cdot\mathbb{E}_{t}\left[\zeta_{t+1}\right]-\zeta_{t}\right)
\end{equation}

Equation (23) ..., where : 
\begin{itemize}
    \item 
\end{itemize}

\subsubsection*{Equation 24}
\begin{equation}
    x_{t}=M\cdot\mathbb{E}_{t}\left[x_{t+1}\right]-\sigma(\hat{r}_{t}-\hat{r}^{n}_{t})
\end{equation}

Equation (24) ..., where : 
\begin{itemize}
    \item 
\end{itemize}

\subsubsection*{Equation 25}
\begin{equation}
    x_{t}=M\cdot\mathbb{E}_{t}\left[x_{t+1}\right]-\sigma(i_{t}-\mathbb{E}_{t}\left[\pi_{t+1}\right]-r^{n}_{t})
\end{equation}

Equation () ..., where : 
\begin{itemize}
    \item 
\end{itemize}

\subsubsection*{Equation 26}
\begin{equation}
    x_{t}=-\sigma\sum_{k\geq 0}{M\cdot \mathbb{E}_{t}\left[\hat{r}_{t+k}-\hat{r}_{t+k}^{n}\right]}
\end{equation}

Equation (26) ..., where : 
\begin{itemize}
    \item 
\end{itemize}

\subsubsection*{Equation 27}
\begin{equation}
    p^{*}_{t}=p_{t}+(1-\beta\theta)\sum_{k=0}^{\infty}\left(\beta\theta\bar{m}\right)^{k}\cdot\mathbb{E}_{t}\left[\pi{t+1}+...+\pi_{t+k}-\mu_{t+k}\right]
\end{equation}

Equation (27) ..., where : 
\begin{itemize}
    \item 
\end{itemize}

\subsection{A Behavioral New Keynesian Model}

\subsubsection*{Equation 28 - Proposition 2, first equation}
\begin{equation}
    x_{t}=M\cdot\mathbb{E}_{t}\left[x_{t+1}\right]-\sigma(i_{t}-\mathbb{E}_{t}\left[\pi_{t+1}\right]-r^{n}_{t})
\end{equation}

Equation (28) ..., where : 
\begin{itemize}
    \item 
\end{itemize}

\subsubsection*{Equation 29 - Proposition 2, second equation}
\begin{equation}
    \pi_{t}=\beta\cdot M^{f} \mathbb{E}_t\left[\pi_{t+1}\right]+\kappa\cdot x_{t}
\end{equation}

Equation (29) ..., where : 
\begin{itemize}
    \item 
\end{itemize}

\subsubsection*{Equation 30}
\begin{equation}
    \begin{cases}
        M=\bar{m} \\
        \sigma=\frac{1}{\gamma R} \\
        M^{f}=\bar{m}\left(\theta+\frac{1-\beta\theta}{1-\beta\theta\bar{m}}(1-\theta)\right)
    \end{cases}
\end{equation}

Equation (30) ..., where : 
\begin{itemize}
    \item 
\end{itemize}

\end{document}