\documentclass{beamer}
\usetheme{metropolis}

\usefonttheme{serif} % default family is serif

\setbeamertemplate{section in toc}[sections numbered]
\setbeamertemplate{subsection in toc}[subsections numbered]

\title{Macroeconomics 2 Presentation}
\subtitle{Article review :\\ Gabaix, Xavier. 2020. "A Behavioral New Keynesian Model." American Economic Review, 110(8): 2271-2327}
\author{GUGELMO CAVALHEIRO DIAS Paulo \\ MITASH Nayanika \\ WANG Shang}
\institute{Sciences Po}
\date{\today}


\newcommand\ReduceFont{\fontsize{10}{7.2}\selectfont}

\begin{document}

\begin{frame}
    \titlepage
\end{frame}

\begin{frame}
    \ReduceFont
    \frametitle{Outline}
    \tableofcontents[hideallsubsections]
\end{frame}

\section{Contextualization}
\begin{frame}
    \tableofcontents[currentsection, hideothersubsections, sections=\value{section}]
\end{frame}

\subsection{Goal of the paper}
\begin{frame}{\subsecname}
    Content of the Goal of the paper.
\end{frame}

\subsection{Literature of the topic}
\begin{frame}
    Content of the the Literature.
\end{frame}

\section{Baseline model of the paper}
\begin{frame}
    \ReduceFont
    \tableofcontents[currentsection, hideallsubsections]
\end{frame}

\begin{frame}
    \tableofcontents[currentsection, hideothersubsections, sections=\value{section}]
\end{frame}

\subsection{Household's Problem}

\subsubsection{Objective function}
\begin{frame}{\subsecname}
    \begin{equation}
        \label{3}
        U = \mathbb{E}\left[\sum_{t=0}^{\infty}\beta^{t}u(c_t,N_t)\right]
    \end{equation}
    With
    \begin{equation*}
    u(c_t,N_t) = \frac{c^{1-\gamma}-1}{1-\gamma}-\frac{N^{1+\phi}}{1+\phi}
    \end{equation*}
    So we have the following objective function of the houshold : 
    \begin{equation*}
        U = \mathbb{E}\left[\sum_{t=0}^{\infty}\beta^{t}\left(\frac{c^{1-\gamma}-1}{1-\gamma}-\frac{N^{1+\phi}}{1+\phi}\right)\right]
    \end{equation*}
\end{frame}

\subsubsection{Constraint}
\begin{frame}{\subsecname}
    \begin{equation}\label{4}
        k_{t+1}=(1+r_t)(k_t-c_t+y_t)
    \end{equation}
\end{frame}

\subsubsection{Rest...}

\subsection{Firms}

\subsection{Solution}

\subsection{Synthesis Of A Behavioral New Keynesian Model}

\subsection{Calibration}

\section{Consequences}
\begin{frame}
    \ReduceFont
    \tableofcontents[currentsection, hideallsubsections]
\end{frame}

\begin{frame}
    \tableofcontents[currentsection, hideothersubsections, sections=\value{section}]
\end{frame}

\section{Implications for monetary policy}
\begin{frame}
    \ReduceFont
    \tableofcontents[currentsection, hideallsubsections]
\end{frame}

\begin{frame}
    \tableofcontents[currentsection, hideothersubsections, sections=\value{section}]
\end{frame}

\section{Implications for fiscal policy}
\begin{frame}
    \ReduceFont
    \tableofcontents[currentsection, hideallsubsections]
\end{frame}

\begin{frame}
    \tableofcontents[currentsection, hideothersubsections, sections=\value{section}]
\end{frame}

\section{Behavioral Enrichments of the Model}
\begin{frame}
    \ReduceFont
    \tableofcontents[currentsection, hideallsubsections]
\end{frame}

\begin{frame}
    \tableofcontents[currentsection, hideothersubsections, sections=\value{section}]
\end{frame}

\subsection{Term Structure of Consumer Attention}
\begin{frame}{\subsecname}
    \begin{equation} \tag{49}
        \begin{split}
            k_{t+1}= &\  \textbf{G}^{k,BR}(c_{t},N_{t},k_{t},\textbf{X}_{t}) \\
            & := (1+\bar{r}+\hat{r}^{BR}(\textbf{X}_t))(k_{t}+\bar{y}+\hat{y}^{BR}(N_{t},\textbf{X}_t)-c_{t})
        \end{split}
    \end{equation}
    \begin{equation}\tag{50}
        \begin{cases}
            \hat{r}^{BR} = m_{r} \hat{r}(\textbf{X}_{t}) \\
            \hat{y}^{BR}(N_{t},\textbf{X}_{t}) = m_{y}\hat{y}(\textbf{X}_{t})+\omega(\textbf{X}_{t})(N_{t}-N_{t}\textbf{X}_{t})
        \end{cases}
    \end{equation}
    \begin{equation}\tag{51}
        \begin{cases}
            \mathbb{E}_{t}^{BR}\left[\hat{r}^{BR}(\textbf{X}_{t+k})\right]=m_{r}\bar{m}^{k}\mathbb{E}_{t}\left[\hat{r}(\textbf{X}_{t+k})\right] \\
            \mathbb{E}_{t}^{BR}\left[\hat{y}^{BR}(\textbf{X}_{t+k})\right]=m_{r}\bar{m}^{k}\mathbb{E}_{t}\left[\hat{y}(\textbf{X}_{t+k})\right]
        \end{cases}
    \end{equation}
\end{frame}

\begin{frame}{\subsecname}
    \begin{equation}\tag{52}
        \hat{c}_{t}=\mathbb{E}_{t}\left[\sum_{\tau\geq t}\frac{\bar{m}^{\tau-t}}{R^{\tau-t}}\left(b_{r}m_{r}\hat{r}(\textbf{X}_{\tau})+m_{Y}\frac{\bar{r}}{R}\hat{y}(\textbf{X}_{\tau})\right)\right]
    \end{equation}
    \begin{equation}\tag{53}
        \frac{\Delta^{GE}}{\Delta^{\text{direct}}}=R^{\tau+1}
    \end{equation}
    \begin{equation}\tag{54}
        \frac{\Delta^{GE}}{\Delta^{\text{direct}}}=\left(\frac{R}{R-rm_{Y}}\right)^{\tau+1}\in\left[1, R^{\tau+1}\right]
    \end{equation}
\end{frame}


\subsection{Flattening of the Phillips Curve via Imperfect Firm Attention}
\begin{frame}{\subsecname}
    \begin{equation}\tag{55}
        \varv^{BR}(q_{it},(\textbf{X}_{\tau})):=\varv^{0}\left(q_{it}-m^{f}_{\pi}\Pi(\textbf{X}_{\tau}),m^{f}_{x}\mu(\textbf{X}_{\tau}),c(\textbf{X}_{\tau})\right)
    \end{equation}
    \begin{equation}\tag{56}
        \max_{q_{it}}{\mathbb{E}_{t}^{BR}\left[\sum_{\tau=t}^{\infty}\left(\beta\theta\right)^{\tau-t}\frac{c(\textbf{X}_{\tau})^{-\gamma}}{c(\textbf{X}_{\tau})^{-\gamma}}\varv^{BR}(q_{it},\textbf{X}_{\tau})\right]}
    \end{equation}
    \begin{equation}\tag{57}
        p^{*}_{t}=p_{t}+(1-\beta\theta)\sum^{\infty}_{k=0} \left(\beta\theta\bar{m}\right)^{k}\mathbb{E}_{t}\left[m^{f}_{\pi}(\pi_{t+1}+...+\pi_{t+k})-m^{f}_{x}\mu_{t+k}\right]
    \end{equation}
    \begin{equation}\tag{58}
        \kappa = m^{f}_{x}\bar{\kappa}
    \end{equation}
\end{frame}

\subsection{Nonconstant Trend Inflation and Neo- Fisherian Paradoxes}
\begin{frame}{\subsecname}
    \begin{equation}\tag{59}
        \pi^{d}_{t}=(1-\zeta)\bar{\pi}_{t}+\zeta\bar{\pi}_{t}^{CB}
    \end{equation}
    \begin{equation}\tag{60}
        x_{t}=M\mathbb{E}_{t}\left[x_{t+1}\right]-\sigma\left(i_{t}-\mathbb{E}_{t}\left[\pi_{t+1}\right]-r^{n}_{t}\right)
    \end{equation}
    \begin{equation}\tag{61}
        \pi_{t}=\beta\cdot M^{f} \mathbb{E}_t\left[\hat{\pi}_{t+1}\right]+\kappa\cdot x_{t}
    \end{equation}
    \begin{equation}\tag{62}
        \phi_{\pi}+\zeta \frac{(1-\beta M^{f})}{\kappa}\phi_{x}+\zeta\frac{(1-\beta M^{f})(1-M)}{\kappa \sigma}>1
    \end{equation}
\end{frame}

\section{Discussion of the Behavioral Assumptions}
\begin{frame}
    \ReduceFont
    \tableofcontents[currentsection, hideallsubsections]
\end{frame}

\begin{frame}
    \begin{itemize}
        \item Theoretical Microfoundation
        \item Lucas Critique
        \item Long-Run Learning
        \item Parsimony and New Degrees of Freedom
        \item Reasonable Variants
    \end{itemize}
\end{frame}

\section{Conclusion}
\begin{frame}
    \ReduceFont
    \tableofcontents[currentsection, hideallsubsections]
\end{frame}

\begin{frame}
    \tableofcontents[currentsection, hideothersubsections, sections=\value{section}]
\end{frame}

\section{Limits and Critics}
\begin{frame}
    \ReduceFont
    \tableofcontents[currentsection, hideallsubsections]
\end{frame}

\begin{frame}
    \tableofcontents[currentsection, hideothersubsections, sections=\value{section}]
\end{frame}



\end{document}

