\documentclass{beamer}
\usepackage{amsmath}
\usepackage{amssymb}
\usepackage{bm}
\usepackage{hyperref}
\usepackage{csquotes}

\DeclareSymbolFont{matha}{OML}{txmi}{m}{it}% txfonts
\DeclareMathSymbol{\varv}{\mathord}{matha}{118}
\usetheme{metropolis}

\usefonttheme{serif} % default family is serif

\setbeamertemplate{section in toc}[sections numbered]
\setbeamertemplate{subsection in toc}[subsections numbered]

\title{Macroeconomics 2 Presentation}
\subtitle{Article review :\\ Gabaix, Xavier. 2020. "A Behavioral New Keynesian Model." American Economic Review, 110(8): 2271-2327}
\author{GUGELMO CAVALHEIRO DIAS Paulo \\ MITASH Nayanika \\ WANG Shang}
\institute{Sciences Po}
\date{\today}

\newcommand\ReduceFont{\fontsize{10}{7.2}\selectfont}

\begin{document}

\begin{frame}
    \titlepage
\end{frame}

\begin{frame}
    \ReduceFont
    \frametitle{Outline}
    \tableofcontents[hideallsubsections]
\end{frame}

\section{Contextualization}
\begin{frame}
    \tableofcontents[currentsection, hideothersubsections, sections=\value{section}]
\end{frame}

\subsection{Goal of the paper}
\begin{frame}{\subsecname}
    Content of the Goal of the paper.
\end{frame}

\subsection{Literature of the topic}
\begin{frame}
    Content of the the Literature.
\end{frame}

\section{Baseline model of the paper}
\begin{frame}
    \ReduceFont
    \tableofcontents[currentsection, hideallsubsections]
\end{frame}

\begin{frame}
    \tableofcontents[currentsection, hideothersubsections, sections=\value{section}]
\end{frame}

\subsection{Household's Problem}

\subsection{Firms}

\subsection{Solution}

\subsection{Synthesis Of A Behavioral New Keynesian Model}

\subsection{Calibration}

\section{Consequences}
\begin{frame}
    \ReduceFont
    \tableofcontents[currentsection, hideallsubsections]
\end{frame}

\begin{frame}
    \tableofcontents[currentsection, hideothersubsections, sections=\value{section}]
\end{frame}

\section{Implications for monetary policy}
\begin{frame}
    \ReduceFont
    \tableofcontents[currentsection, hideallsubsections]
\end{frame}

\begin{frame}
    \tableofcontents[currentsection, hideothersubsections, sections=\value{section}]
\end{frame}

\section{Implications for fiscal policy}
\begin{frame}
    \ReduceFont
    \tableofcontents[currentsection, hideallsubsections]
\end{frame}

\begin{frame}
    \tableofcontents[currentsection, hideothersubsections, sections=\value{section}]
\end{frame}

\section{Behavioral Enrichments of the Model}

\begin{frame}
    \ReduceFont
    \tableofcontents[currentsection, hideallsubsections]
\end{frame}

\begin{frame}
    \tableofcontents[currentsection, hideothersubsections, sections=\value{section}]
\end{frame}

\subsection{Term Structure of Consumer Attention}
\begin{frame}{\subsecname}
    It is plausible that consumers \textbf{do not equally pay attention to all economic variables}, even in the present. 
    We could therefore introduce attention discount factors that are variable specific, yielding \textbf{perceived variables} under Bounded Rationality : 
    \begin{itemize}
        \item $\hat{r}^{BR}$ the perceived interest rate under bounded rationality
        \item $\hat{y}^{BR}$ the perceived income under bounded rationality
    \end{itemize}
    Prior to this, consumers perceived perfectly variables at the current period, now, they do not anymore. \\
    \hfill \linebreak
    Directly affects the consumer maximisation program. 
\end{frame}
    
\begin{frame}{\subsecname}
    The law of motion of the personal wealth of the consumer becomes thus a \textbf{perceived law of motion}
    \begin{equation}\tag{6}
        \begin{split}
            k_{t+1}= &\ G^{k}(c_{t},N_{t}, k_{t}, \bm{X}_{t}) \\ 
            & := (1+\bar{r}+\hat{r}(\bm{X}_{t}))(k_{t}+\bar{y}+\hat{y}(N_{t},\bm{X}t)-c_{t})
        \end{split}
    \end{equation}
    Turns into :    
    \begin{equation} \tag{49}
        \begin{split}
            k_{t+1}= &\  \textbf{G}^{k,BR}(c_{t},N_{t},k_{t},\textbf{X}_{t}) \\
            & := (1+\bar{r}+\hat{r}^{BR}(\textbf{X}_t))(k_{t}+\bar{y}+\hat{y}^{BR}(N_{t},\textbf{X}_t)-c_{t})
        \end{split}
    \end{equation}
    \hfill \linebreak
    What could be functional forms of $\hat{r}^{BR}$ and $\hat{y}^{BR}$ ?
\end{frame}

\begin{frame}{\subsecname}
    The perceived values of interest rate and income are defined such that :
    \begin{equation}\tag{50}
        \begin{cases}
            \hat{r}^{BR} = m_{r}\cdot\hat{r}(\textbf{X}_{t}) \\
            \hat{y}^{BR}(N_{t},\textbf{X}_{t}) = m_{y}\cdot\hat{y}(\textbf{X}_{t})+\omega(\textbf{X}_{t})(N_{t}-N_{t}(\textbf{X}_{t}))
        \end{cases}
    \end{equation}
    Equation (50) is a possible functional form. The main changes are the attention discount factor. \\
    Now, what would be the expectation under behavioral expectation of those perceived values ?
\end{frame}

\begin{frame}{\subsecname}
    Consumers already have a general attention discount factor $\bar{m}$, from Lemma 1 in equation (11) :
    \begin{equation*}\tag{11}
        \mathbb{E}_{t}^{BR}\left[z\left(\bm{X}_{t+k}\right)\right]=\bar{m}^{k}\cdot\mathbb{E}_{t}\left[z\left(\bm{X}_{t+k}\right)\right]
    \end{equation*}
    Applied to the perceived interest rate and perceived income, we thus get the \textbf{Lemma 5 (Term Structure of Attention)}:
    \begin{equation}\tag{51}
        \begin{cases}
            \mathbb{E}_{t}^{BR}\left[\hat{r}^{BR}(\textbf{X}_{t+k})\right]=m_{r}\cdot\bar{m}^{k}\cdot\mathbb{E}_{t}\left[\hat{r}(\textbf{X}_{t+k})\right] \\
            \mathbb{E}_{t}^{BR}\left[\hat{y}^{BR}(\textbf{X}_{t+k})\right]=m_{y}\cdot\bar{m}^{k}\cdot\mathbb{E}_{t}\left[\hat{y}(\textbf{X}_{t+k})\right]
        \end{cases}
    \end{equation}
\end{frame}

\begin{frame}{\subsecname}
    What are consequences of this enriched attention structure term ? \\
    When we solve for consumption, we get \textbf{Proposition 8} (Behavioral Consumption Function) :
    \begin{equation}\tag{52}
        \hat{c}_{t}=\mathbb{E}_{t}\left[\sum_{\tau\geq t}\frac{\bar{m}^{\tau-t}}{R^{\tau-t}}\left(b_{r}m_{r}\hat{r}(\textbf{X}_{\tau})+m_{Y}\frac{\bar{r}}{R}\hat{y}(\textbf{X}_{\tau})\right)\right]
    \end{equation}
    With :
    \begin{columns}
        \begin{column}{0.5\textwidth}
            \begin{equation*}
                \begin{cases}
                    c_{t}=c_{t}^{d}+\hat{c}_{t} \\ 
                    c^{d}_{t} = \bar{y} + b_{k}\cdot k_{t} \\
                    b_{k}:=\frac{\bar{r}}{R}\cdot\frac{\phi}{\phi+\gamma} \\ 
                \end{cases}
            \end{equation*}
        \end{column}
        \begin{column}{0.5\textwidth}  %%<--- here
             \begin{equation*}
                \begin{cases}
                    m_{Y}=\frac{\phi\cdot m_{y}+\gamma}{\phi+\gamma} \\
                    b_{r}:=-\frac{1}{\gamma\cdot R^{2}}
                \end{cases}
             \end{equation*}
        \end{column}
    \end{columns}
\end{frame}

\begin{frame}{\subsecname}
    Interest rate has \textbf{direct} and \textbf{indirect} effects on consumption. \\ 
    For a consumer, a decrease in future interest rate : 
    \begin{itemize}
        \item increases their present consumption, because it is more profitable to consume right now (direct effect)
        \item increases other consumers future consumption, increasing their future income, increasing their current consumption (indirect effect)
    \end{itemize}
    Therefore, the aggregate consumption multiplies the positive effect on consumption of a decrease in future interest rate. 
    What does this behavioral model imply for this multiplicator ?
\end{frame}

\begin{frame}{\subsecname}
    In the \textbf{rational consumer} case : \\

    If we derive from equation (52), we get the direct effect : 
    \begin{equation*}
        \Delta^{\text{direct}}:=\frac{\partial \hat{c}_{0}}{\partial \hat{r}_{\tau}}\bigg\rvert_{(y_{t})_{t\geq0 \text{ held constant}}} = -\alpha\cdot \frac{1}{R^{\tau}}
    \end{equation*}
    
    If we derive from equation (26), we get the indirect effect :
    \begin{equation*}
        \Delta^{GE}:=\frac{\partial \hat{c}_{0}}{\partial \hat{r}_{\tau}}=-\alpha R 
    \end{equation*}

    Put together : 
    \begin{equation}\tag{53}
        \frac{\Delta^{GE}}{\Delta^{\text{direct}}}=R^{\tau+1}
    \end{equation}
\end{frame}

\begin{frame}{\subsecname}
    In the \textbf{behavioral consumer} case : \\

    If we derive from equation (52), we get the direct effect : 
    \begin{equation*}
        \Delta^{\text{direct}}:=\frac{\partial \hat{c}_{0}}{\partial \hat{r}_{\tau}}\bigg\rvert_{(y_{t})_{t\geq0 \text{ held constant}}} = -\alpha\cdot m_{r}\cdot\bar{m}^{\tau}\frac{1}{R^{\tau}}
    \end{equation*}

    If we derive from equation (26), we get the indirect effect :
    \begin{equation*}
        \Delta^{GE}:=\frac{\partial \hat{c}_{0}}{\partial \hat{r}_{\tau}}=-\alpha m_{r}\cdot M^{\tau} \frac{R}{R-r\cdot m_{Y}}R 
    \end{equation*}

    Put together : 
    \begin{equation}\tag{54}
        \frac{\Delta^{GE}}{\Delta^{\text{direct}}}=\left(\frac{R}{R-rm_{Y}}\right)^{\tau+1}\in\left[1, R^{\tau+1}\right]
    \end{equation}
\end{frame}

\begin{frame}{\subsecname}
    In a behavioral framework, the multiplicative effect is dampened by bounded rationality. \\

    An attention discount factor that is variable specific allows to explain why forward guidance is not as strong as what theory predicts. \\ 

    What about variable specific attention deficiency for firms now ?
\end{frame}

\subsection{Flattening of the Phillips Curve via Imperfect Firm Attention}
\begin{frame}{\subsecname}
    If we introduce variable specific inattention for firms, equation (15), defining the real profit of the firm : 
    \begin{equation}\tag{15}
        \varv\left(q_{it},\bm{X}{\tau}\right):=\varv^{0}\left(q_{it}-\Pi(\bm{X}{\tau}),\mu(\bm{X}{\tau}),c(\bm{X}_{\tau})\right)
    \end{equation}
    Turns into a perceived real profit of the firm :
    \begin{equation}\tag{55}
        \varv^{BR}(q_{it},(\textbf{X}_{\tau})):=\varv^{0}\left(q_{it}-m^{f}_{\pi}\cdot\Pi(\textbf{X}_{\tau}),m^{f}_{x}\cdot\mu(\textbf{X}_{\tau}),c(\textbf{X}_{\tau})\right)
    \end{equation}
    Where : 
    \begin{itemize}
        \item $m^{f}_{\pi}$ is the attention deficit to inflation 
        \item $m^{f}_{x}$ is the attention deficit to marginal cost
    \end{itemize}
\end{frame}

\begin{frame}{\subsecname}
    The maximisation program of equation (16) :
    \begin{equation}\tag{16}
        \max_{q_{it}} \mathbb{E}_{t}\left[\sum_{\tau=t}^{\infty}(\beta\theta)^{\tau-t} \frac{c\left(\bm{X_{\tau}}\right)^{-\gamma}}{\left(\bm{X_{t}}\right)^{-\gamma}} \varv\left(q_{it},\bm{X}{\tau}\right)\right]
    \end{equation}
    turns into :
    \begin{equation}\tag{56}
        \max_{q_{it}}{\mathbb{E}_{t}^{BR}\left[\sum_{\tau=t}^{\infty}\left(\beta\theta\right)^{\tau-t}\frac{c(\textbf{X}_{\tau})^{-\gamma}}{c(\textbf{X}_{t})^{-\gamma}}\varv^{BR}(q_{it},\textbf{X}_{\tau})\right]}
    \end{equation}
\end{frame}

\begin{frame}{\subsecname}
    Solving it yields :
    \begin{equation}\tag{57}
        \begin{split}
            & p^{*}_{t} = p_{t}+(1-\beta\theta) \cdot \\
            & \sum^{\infty}_{k=0} \left(\beta\theta\bar{m}\right)^{k}\mathbb{E}_{t}\left[m^{f}_{\pi}(\pi_{t+1}+...+\pi_{t+k})-m^{f}_{x}\mu_{t+k}\right]
        \end{split}
    \end{equation}
    In comparison, we had in the baseline model : 
    \begin{equation*}\tag{27}
        p^{*}_{t}=p_{t}+(1-\beta\theta)\sum_{k=0}^{\infty}\left(\beta\theta\bar{m}\right)^{k}\cdot\mathbb{E}_{t}\left[\pi_{t+1}+...+\pi_{t+k}-\mu_{t+k}\right]
    \end{equation*}
\end{frame}

\begin{frame}{\subsecname}
    We also get : 
    \begin{equation*}
        M^{f}=\bar{m}\left(\theta+m^{f}_{\pi}\cdot(1-\theta)\cdot\frac{1-\beta\cdot\theta}{1-\beta\cdot\theta\cdot\bar{m}}\right)\in\left[0,1\right]
    \end{equation*}
    \begin{equation}\tag{58}
        \kappa = m^{f}_{x}\bar{\kappa}
    \end{equation}

    Where 
    \begin{itemize}
        \item $M^{f}$ is the general attention factor of the firm
        \item $m^{f}_{x}$ is the attention deficiency to the output gap
        \item $\kappa=m_{x}^{f}\cdot\bar{\kappa}$, is the perceived value of the importance of outputgap on inflation
    \end{itemize}
\end{frame}

\begin{frame}{\subsecname}
    If we solve the Phillips curve, the equation (29) : 
    \begin{equation}\tag{29}
        \pi_{t}=\beta\cdot M^{f}\cdot\mathbb{E}t\left[\pi_{t+1}\right]+\kappa\cdot x_{t}
    \end{equation}
    Turns into a Phillips Curve with Behavioral Firms, allowing for imperfect attention to inflation and costs (\textbf{Proposition 10}) :
    \begin{equation*}
        \pi_{t}=\beta\cdot\bar{m}\left(\theta+m^{f}_{\pi}\cdot(1-\theta)\cdot\frac{1-\beta\cdot\theta}{1-\beta\cdot\theta\cdot\bar{m}}\right)\cdot\mathbb{E}t\left[\pi_{t+1}\right]+m_{x}^{f}\cdot\bar{\kappa}\cdot x_{t}
    \end{equation*}
\end{frame}

\subsection{Nonconstant Trend Inflation and Neo- Fisherian Paradoxes}
\begin{frame}{\subsecname}
    Now, let's change the way we consider inflation. Instead of a steady state, agents perceive a default value.
    The article proposes the following functional form :
    \begin{equation}\tag{59}
        \pi^{d}_{t}=(1-\zeta)\bar{\pi}_{t}+\zeta\bar{\pi}_{t}^{CB}
    \end{equation}
    The IS curve is unchanged, and is the same as equation (28) :
    \begin{equation}\tag{60}
        x_{t}=M\mathbb{E}_{t}\left[x_{t+1}\right]-\sigma\left(i_{t}-\mathbb{E}_{t}\left[\pi_{t+1}\right]-r^{n}_{t}\right)
    \end{equation}
\end{frame}

\begin{frame}{\subsecname}
    The Phillips curve of equation (29) :
    \begin{equation}\tag{29}
        \pi_{t}=\beta\cdot M^{f} \mathbb{E}_{t}\left[\pi_{t+1}\right]+\kappa\cdot x_{t}
    \end{equation}
    Turns into :
    \begin{equation}\tag{61}
        \hat{\pi}_{t}=\beta\cdot M^{f}\cdot\mathbb{E}_t\left[\hat{\pi}_{t+1}\right]+\kappa\cdot x_{t}
    \end{equation}
    The only difference is through $\hat{\pi}_{t}$.
\end{frame}

\begin{frame}{\subsecname}
    Finally, the equilbrium condition changes through $\zeta$ the weight given to the Central Bank declaration. 
    Equation (34) in the Baseline model :
    \begin{equation}\tag{34}
        \phi_{\pi}+\frac{(1-\beta M^{f})}{\kappa}\phi_{x}+\frac{(1-\beta M^{f})(1-M)}{\kappa\sigma}>1
    \end{equation}
    Turns into :
    \begin{equation}\tag{62}
        \phi_{\pi}+\zeta \frac{(1-\beta M^{f})}{\kappa}\phi_{x}+\zeta\frac{(1-\beta M^{f})(1-M)}{\kappa \sigma}>1
    \end{equation}
    The term $\zeta$ can be considered as the weight of central bank guidance.
    \begin{itemize}
        \item What if it is 0 ?
        \item What are the consequences for Central Bankers ?
    \end{itemize}
\end{frame}


\section{Discussion of the Behavioral Assumptions}
\begin{frame}
    \ReduceFont
    \tableofcontents[currentsection, hideallsubsections]
\end{frame}

\begin{frame}
    \begin{itemize}
        \item Theoretical Microfoundation
        \item Lucas Critique
        \item Long-Run Learning
        \item Parsimony and New Degrees of Freedom
        \item Reasonable Variants
    \end{itemize}
\end{frame}


\section{Conclusion}
\begin{frame}
    \ReduceFont
    \tableofcontents[currentsection, hideallsubsections]
\end{frame}

\begin{frame}
    \tableofcontents[currentsection, hideothersubsections, sections=\value{section}]
\end{frame}

\section{Limits and Critics}
\begin{frame}
    \ReduceFont
    \tableofcontents[currentsection, hideallsubsections]
\end{frame}

\begin{frame}
    \tableofcontents[currentsection, hideothersubsections, sections=\value{section}]
\end{frame}



\end{document}

